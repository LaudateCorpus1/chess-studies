%% skiminki's computer chess studies
%% Copyright (C) 2019 Sami Kiminki
%%
%% This program is free software: you can redistribute it and/or modify
%% it under the terms of the GNU General Public License as published by
%% the Free Software Foundation, either version 3 of the License, or
%% (at your option) any later version.
%%
%% This program is distributed in the hope that it will be useful,
%% but WITHOUT ANY WARRANTY; without even the implied warranty of
%% MERCHANTABILITY or FITNESS FOR A PARTICULAR PURPOSE.  See the
%% GNU General Public License for more details.
%%
%% You should have received a copy of the GNU General Public License
%% along with this program. If not, see <http://www.gnu.org/licenses/>.

\documentclass[a4paper,11pt,twocolumn]{report}
\usepackage{times}
\usepackage{textcomp}
\usepackage[protrusion=true,expansion=true]{microtype}
\usepackage[LSBC1,LSBC2,LSBC4,T1]{fontenc}
\usepackage[utf8]{inputenc}
\usepackage{pgf}
\usepackage{xcolor}
\usepackage[skaknew]{chessboard,skak}
\usepackage[shortlabels]{enumitem}
\font\sknf=SkakNew-Figurine
\font\sknfbx=SkakNew-FigurineBold
\font\skndia=SkakNew-DiagramT

\title{Instructive Computer Games of Chess:\\What we can learn from the beasts?}

\newcommand{\abbeg}[0]{\emph{e.g.}\@}
\newcommand{\abbie}[0]{\emph{i.e.}\@}
\newcommand{\abbetal}[0]{\emph{et al.}\@}

\newcommand{\chessgame}[5]{%
  \begin{center}%
     \section{\textbf{#1} -- \textbf{#2}}% participants
     \textsf{#3}%                          event

     \small{\textsc{#4}}%                  opening

     \small{\emph{#5}}%                    date
  \end{center}%
  \styleC\newgame%
}

% notes: styleC is the nice move pair per line; styleB is the condensed format.

\begin{document}
\maketitle

\chapter{Strategic Advantage}

%% skiminki's computer chess studies
%% Copyright (C) 2019 Sami Kiminki
%%
%% This program is free software: you can redistribute it and/or modify
%% it under the terms of the GNU General Public License as published by
%% the Free Software Foundation, either version 3 of the License, or
%% (at your option) any later version.
%%
%% This program is distributed in the hope that it will be useful,
%% but WITHOUT ANY WARRANTY; without even the implied warranty of
%% MERCHANTABILITY or FITNESS FOR A PARTICULAR PURPOSE.  See the
%% GNU General Public License for more details.
%%
%% You should have received a copy of the GNU General Public License
%% along with this program. If not, see <http://www.gnu.org/licenses/>.

\chessgame{LCZero v0.21.1-nT40.T6.532}{Bluefish Dev}%
          {TCEC S15 Bonus: Bluefish vs Leela Jhorthos, Game 2}%
          {E05 Catalan, Open, Classical line}%
          {2019-04-28}

This game is a remarkable example of taking and keeping strategic
advantage. After the opening phase, it feels that white dictated the
direction and black reacted, with black never having a say for where
the game would be heading to.

Following concepts exemplified:
\begin{enumerate}
\item Restricting opponent pieces for strategic advantage
\item Shifting focus from one side of the board to another, to take
advantage of better piece mobility
\end{enumerate}

The game started from the regular starting position.

\mainline{1. d4 Nf6 2. c4 e6 3. g3}

The Catalan opening seems to be a favoured Queen's Gambit opening for
the current NN engines.

\mainline{3... d5 4. Bg2 Be7 5. Nf3 O-O 6. O-O dxc4
  7. Qc2 b5?!}

\chessboard

Typical play here is \movecomment{7... a6,} instead. Trying to hold on
to c4 allows white to get a significant amount for activity for the
pawn.

\mainline{8. a4 b4 9. Nfd2}

While \movecomment{9. Nbd2} looks perhaps a bit more natural and is
slightly more popular in the Lichess GM
database, \movecomment{9. Nfd2} has better statistics for white. The
merit of the move in the game is that it unblocks the bishop's vision,
adding pressure towards the a8 square.

Black has here three options to protect the
rook. \movecomment{9... Nd5} gives up the pawn back and allows black
to develop the queenside. \movecomment{9... b3 10. Qxc4 Ba6 11. Qxb3}
and either \movecomment{11... c6} and black will get one pawn back, or
\movecomment{11... Nd5} for trading one pawn for better development. A
third and the most popular option was played in the game, postponing
the resolution of the queenside development.

\mainline{9... c6 10. Nxc4 Qxd4 11. Rd1 Qc5 12. Be3 Qh5 13. Nbd2 Nd5!?\novelty}

\chessboard

\movecomment{13... Ng4} was the most popular move. This move has the
merit over the played move that it effectively forces
\movecomment{14. Nf3} or \movecomment{14. Nf1} in order to stop the
mate threat, unless white decides to weaken the king-side pawn
structure by \wmove{h3} or \wmove{h4.} This deflects the knight from
the d2 square.

\movecomment{13... Nd5} would make more sense if the purpose was to
block the g2-bishop eyeing towards the black's queen-side
corner. However, the intention was to trade the knight with the
bishop.

\mainline{14. Nb3 Nxe3?! 15. Nxe3}

\chessboard[pgfstyle=straightmove,
  color=green, markmoves={d1-d8},
  color=red, markmoves={c2-c6, g2-c6},
  pgfstyle=knightmove, markmoves={b3-a5, a5-c6}]

Now the small subtlety of playing \movecomment{13... Nd5} instead of
\movecomment{13... Ng5} becomes obvious: black is going to have some
serious questions to answer on developing the queen side, as the c6
pawn is starting to become a liability. Further, white has gained the
d-file, thwarting development ideas such as \wmove{Bd7} with
\wmove{Na6} for now. So, black goes with another typical development
idea in Catalan: push the a-pawn, play \wmove{Ra7}, and then untangle
with moves such as \wmove{Bg7} and \wmove{c5}.

\mainline{15... a6}

Now move \wmove{Ra7} is enabled. The move \movecomment{15... a6} over
the move \movecomment{15... a5} has the following two benefits: (1)
The a-pawn will not become a target for a later
\wmove{Nxa5,} and (2) the pawn controls the b5 square after \wmove{c5.}

\mainline{16. Nc4 Ra7 17. Rac1 c5 18. Nba5}

\chessboard[color=red!30, pgfstyle=color,
  colorbackfields={a8,b7,c6,d6,b6},
  color=yellow!50, colorbackfields={d7},
  color=blue, pgfstyle=straightmove, markmoves={c4-b6, c4-d6},
  color=red, pgfstyle=straightmove, markmoves={c2-c5}]

The b4-pawn is now finally protected, but black's problems are still
far from over. The black queen-side is a minefield due to white's
control, and the d7-square is a bottleneck for black
development. Further, should the e7-bishop move, white is ready to
jump the c4-knight exposing the threat to take the c5-pawn.

Therefore, it is no wonder that it is already getting difficult to
find any useful moves for black:

\begin{enumerate}[label=(\alph*)]
\item \movecomment{18... Rc7} does not help development, since the c6
square is already attacked twice by white. Similarly, a natural
move \movecomment{18... Bg7} cannot be played.
\item \movecomment{18... Qg6} only asks white to make another
developing move such as \movecomment{19. Rd3/e4/Be4}.
\item \movecomment{18... Rd1} allows white to practically
force \movecomment{19. Rxd8+ Bxd8 20. Rd1 Bxa5 21. Nxa5 Qg5 22. Rd6
Rc7 23. Qd1 Nd7 24. Rc6 Rxc6 25. Nxc6.} The
\movecomment{26... c4} move here would be met with \movecomment{27. Qd6}
threatening to either win a piece by \wmove{Ne7+} or a pawn
by \wmove{Qxb4} while still keeping black's queenside development
cumbersome.
\item \movecomment{18... Bd7} with a possible
continuation \movecomment{19. b3 Rc7 20. Qd3 Rd8 21. Qe3 Be8 22. Qf4
Rcd7 23. Rxd7 Bxd7 24. e3} threatening \wmove{Qc7}, for instance.
\end{enumerate}

Black simply does not seem to be able to find any useful counterplay,
so black decided to play a semi-waiting queen move, reinforcing the d8
square.

\mainline{18... Qg5 19. h4! Qf6}
\chessboard

As black is unable to make progress, white is now starting to improve
her position. The move \wmove{h4} not only restricts the queen, but it
also signals white's intention to shift the play in the
king-side. Often such ideas are useful when one side has better access
to squares.

\mainline{20. Qe4 Rc7 21. Rd3 g5}

Black was soon forced to do something. If black would continue to play
waiting moves, then white would soon break black's position. An
example line \variation{21... Kh8 22. b3 Kg8 23. Rcd1 Kh8 24. Rf3 Qh6
25. Qe5} with unparriable threats. For example: \movecomment{25...
Rd7 26. Rfd3 Rxd3 27. Rxd3 Qc1+ 28. Bf1 Nd7 29. Qc7 Bf6 30. Nc6 a5
31. Na7 Ba6 32. Rxd7} with no hope for black.

\mainline{22. h5! Qg7 23. h6!!}
\chessboard

With the h-pawn moves, white continues to strangle black's position,
and begins to threaten a king-side attack. Note that the h6-pawn
cannot be taken as black queen must be ready to stop \wmove{Qe5}, as
that would win a piece due to the skewer (\movecomment{23... Qxh6??
24. Qe5 Rd7 25. Qxb8).}

\mainline{23... Qf6 24. b3}

Solidifying move by white and preventing potential tactics
by \bmove{b3}. Note that often with such moves, white needs to pay
attention not to trap pieces, as b3 was the natural escape square for
the a5-knight. However, here white could simply move the c4-knight
somewhere to provide another escape square, should the need arise.

\mainline{24... Bd7 25. Qe3 g4 26. Rcd1 Be8 27. Kh2}
\chessboard

The king move is a subtle move in this complex position. Black is now
almost in a Zugswang where it would be preferable not to make a move
at all. The king move is also preparatory to avoid tempo loss
by \wmove{Kh8/Rg8+} in some variations after both the g-pawns have
moved to other files, as well as to prepare Rh1 to protect the h6-pawn
with possible rook lift ideas. Finally, \wmove{Kh2}
prepares \wmove{Kg3} to attack the g4-pawn should the opportunity
arise. Let us review some of black's choices:

\begin{enumerate}[label=(\alph*)]
\item \movecomment{27... Qf5} would essentially transpose to the game
continuation after \movecomment{28. Qf4}
\item \movecomment{27... Kh8 28. Ne5 Qf5 29. Qf4 Qxf4 30. gxf4} and
white maintains the advantage with a continuation such
as \movecomment{30... f6 31. Nec4 Bg6 32. R3d2 Rg8 33. Kg3 Bf8 34. e3
Bxh6 35. Rd8 Re7 36. Nb7 Rxd8 37. Rxd8+ Re8 38. Rxe8+ Bxe8 39. Nxc5.}
Note that a move \movecomment{29... Qh5+} would not save the day,
since \movecomment{30. Kg1 Qf5 31. Qxf5 exf5} would just leave white
with a better pawn structure with otherwise similar prospects.
\item \movecomment{27... Bd7 28. Ne5 Be8 29. Nxg4} and white simply
wins a free pawn.
\item \movecomment{27... Ra7} followed by, \emph{e.g.}, \movecomment{28. Qe4 Qxh6+
29. Kg1 Rc7 30. Qxg4+ Qg5 31. Qxg5+ Bxg5 32. Rd6} and white gets a
strong foothold into black's territory, renewing the question on the
queen-side weaknesses. Also, the exchange sacrifice
after \movecomment{28. Rd6 Bxd6 29. Rxd6 Rd8 30. Rxd8 Qxd8 31. Qe5 f6
32. Qxe6+} looks interesting with good compensation, although probably
not necessary to take a risk.
\item \movecomment{27... Ra7 28. Nb7 Qg5 29. Qxg5+ Bxg5 30. Nxc5 Rc7
31. Ne4 Bxh6 32. Nf6+ Kg7 33. Nxe8+ Rxe8 34. Rd4} does not look very
attractive, either.
\end{enumerate}

None of the options look particularly good, so black went with the
straightforward queen exchange.
\mainline{27... Qg5 28. Qf4!}

White does not exchange the queen immediately, but forces better
terms. The exchange on f4 square prevents black to get into the game,
which \variation{28. Qxg5+ Bxg5 29. Rd6 Bxh6 30. Rb6 Bd7 31. Ne5 Bg7}
would have allowed.

\mainline{28... Qxf4}

Black is forced to exchange the queens on white's terms. Note that
curiously, black cannot make a waiting move \movecomment{28... Rc8}
asking again to exchange the queens on the g5 square. The sequence
after \movecomment{29. Qxg5} no longer works because the
move \wmove{Bd7} is no longer possible because the rook had
moved. Also, the intermediate queen check is no good either, because
after \variation{28... Qh5+ 29. Kg1 e5 30. Nxe5 Bg5 31. Qf5 Qxh6
32. Nxg4 Qg6 33. Qe5} black simply drops a piece under white's
threats.

\mainline{29. gxf4}
\chessboard[color=red!30, pgfstyle=color,
  colorbackfields={e5,g5},
  color=black, pgfstyle=straightmove, markmoves={f4-e5, f4-g5}]

The pawn now controls two important squares: e5 provides an anchor
point for a knight, and g5 protects indirectly the h6-pawn by
preventing \wmove{Bg5}.

\mainline{29... Nd7 30. Nc6 Bh4 31. Kg1}
The threat was more important than the execution. Now its time to get
back and protect the f2-pawn.

\mainline{31... Nf6 32. N6e5 Nh5 33. e3 g3}

Black is trying to create counterplay by trying to open the g-file and
repositioning minor pieces in the h-file. However, white has a simple
answer:

\mainline{34. f3!}
\chessboard[color=yellow!50, pgfstyle=color, colorbackfields={h4,h5}]

White simply makes black's own pawn a shield for the potential attack
on the g-file, claiming that the pawn is rather weak. Now the black
minor pieces on the h-file are irreparably mispositioned, and white
can concentrate again on the queen-side. This time black has no answer
to prevent white's penetration with the rooks.

\mainline{34... Be7 35. Nd7! Bxd7 36. Rxd7 Rfc8 37. Nb6! Rxd7 38. Rxd7
Re8}

\chessboard[
    pgfstyle=straightmove, color=green, markmoves={g2-f1, f1-c4, f4-f5},
    color=blue, markmoves={g1-g2, g2-g3},
    color=yellow, markmoves={d7-a7, a7-a5},
    color=red, markmoves={d6-e8},
    pgfstyle=knightmove, color=green, markmoves={b6-c4, c4-d6, d6-f7},
]

White has now multiple pieces for the plan to finish the game:
\begin{enumerate}[label=(\alph*)]
\item The rook can claim the a5-pawn, creating a passed pawn for
white. The cost is some time. (Yellow arrows)
\item The king can march to g2 after the bishop has moved. Then, if
the black knight moves, the g3 pawn can be taken. (Blue arrows)
\item The knight can move to d6 driving \wmove{Re8} away, allowing the
white rook to add additional pressure on f7. The bishop can reposition
to c4, and the pawn can move to f5 adding even more pressure to the
e6/f7 squares. This should allow white to create passed connected
passers on e/f-files. (Green arrows)
\end{enumerate}
Meanwhile, white's h6-pawn has now become weak and cannot be
protected. Some precision is still required.

\mainline{39. Bf1 Nf6 40. Rb7 Nd5 41. Nc4 Bf8 42. Kg2 Bxh6 43. Kxg3}
The king has now claimed the g3 pawn. Note that \movecomment{40. Ra7}
would probably have been slightly more precise, although it does not
matter.

\mainline{43... Bf8 44. e4 Ne7 45. Nd6 Ra8 46. Rd7}

A waiting move to ask black to move again, and to prevent the black to
move \bmove{Rd8} with tempo on the knight.

\mainline{46... a5 47. Bc4}

\chessboard[pgfstyle=straightmove, color=green, markmoves={d7-f7, c4-f7},
            markmoves={d6-f7, f4-f5}]

Now focusing on the f7 pawn with f4 pawn waiting to be moved.

\mainline{47... Nc8 48. Nxc8 Rxc8}

The pawn move f5 is now temporarily discouraged due to
tactics: \variationcurrent{49. f5 Rc7 50. Rxc7 Bd6+ 51. f4 Bxc7.}
However, this would not be a disaster, since
after \movecomment{52. fxe6 fxe6 53. Bxe6+ Kg7 54. e5} the
opposite-colored bishop ending would still be winning for
white. White's connected passed pawns and white king's access to
black's a-b-c pawns are strong enough to win the game. A bishop cannot
usually stop attack on both sides of the board, since one side can
deflect the bishop while the other side decides the game.

However, white did not go into such complications, and simply switched
to taking the a5-pawn first and then resolving the pin by moving the
e4-pawn before pushing the f-pawn.

\mainline{49. Ra7 Bd6 50. Rxa5 Kf8 51. e5}
\chessboard

Black has no good squares for the bishop. \wmove{Bb8} would lose
quickly to \wmove{Ra8} pinning, and then white pushing the a-pawn.

\wmove{51... Be7} would postpone the game by some moves with the
following possible continuation: \movecomment{52. Be7 52. Ra7 Ke8
53. Ba6! Rb8 54. Bb5+ Kf8 55. Bc4.} Note the
move \movecomment{53. Ba6!}. It drives the rook away from the
c8-square, forcing \bmove{Kf8} after the check. The king cannot come
to d8-square after \bmove{Rd8} since it's already occupied, or
after \bmove{Rb8} since then white would play \wmove{Rb7+} with reveal
check, winning the rook.

So, black sacrificed the c5-pawn in order to be able to block the
a-pawn with a bishop.

\mainline{51... Bc7 52. Rxc5 Ke7 53. a5 Kd8 54. a6 Bb6 55. Rb5 Ba7
56. f5}
\chessboard

White is now finally going after the f7-pawn. In opposite-color bishop
endings, it is important to have two passed pawns with some distance
to make it impossible for the bishop to stop both.

\mainline{56... Rb8 57. Ra5}
White could have also exchanged the rooks. The rest is simple
technique.

\mainline{57... exf5 58. Bxf7 Bb6 59. Ra2}
The final pitfall was the natural-looking \wmove{Rb5}. This would have
blundered the win away to simple tactics: \variation{59. Rb5 Bf2+
60. Kxf2 Rxb5} with a disappointing draw. But now the rest is simple
technique.

\mainline{59... Rc8 60. Kf4 h5 61. Kxf5 Rc7 62. Bxh5 Rh7 63. Bg4 Rc7
64. Ra4 Ra7 65. Rxb4 Rxa6}

\chessboard

A quick way to force mate would have been: \variationcurrent{66. Ra4
Ra5 67. Rxa5 Bxa5 68. Ke6 Ke8 69. f4 Bb4 70. f5 Kf8 71. f6 Be1 72. Bh5
Bd2 73. Kd5 Bc3 74. b4 Bxb4 75. e6 Ba3 76. Kc6 Be7 77. fxe7+ Kxe7
78. Bf7 Kf6 79. Kd6 Kf5 80. e7 Ke4 81. e8=Q+ Kd3 82. Qe1 Kc2 83. Qe3
Kb2 84. Qd2+ Kb1 85. Be6 Ka1 86. Qc1\mate}

However, in this game, Leela did not have the distance-to-zero (DTZ)
tablebase files available and was on her own after reaching the
winning position as per the win-draw-loss (WDL) files. As is
characteristic to the current neural network (NN) engines, the endgame
was not concluded quickly. The NN engines are simply interested in
winning the game, not winning the game quickly.

\styleB
\mainline{
66. f4 Bc5 67. Rc4 Bf2 68. Ke4 Rg6 69. Kf3 Bh4 70. b4 Ra6 71. Bf5 Be1
72. b5 Ra3+ 73. Kg4 Rg3+ 74. Kh5 Ba5 75. Ra4 Bb6 76. Bg4 Ke8 77. Kg5
Bd8+ 78. Kf5 Bb6 79. Ra6 Bf2 80. Bh5+ Ke7 81. Re6+ Kd7 82. Be8+ Kc7
83. Rc6+ Kb8 84. Rc4 Ka7 85. e6 Rg7 86. Ke5 Bh4 87. f5 Be7 88. Bg6 Bf8
89. Rc6 Rb7 90. Kf6 Be7+ 91. Kg7 Ba3+ 92. Kh6 Rxb5 93. f6 Bf8+ 94. Kh7
Rb7+ 95. Bf7 Ba3 96. Rc2 Kb6 97. Kg6 Kb5 98. Be8+ Kb6 99. Bd7 Rc7
100. Rxc7 Kxc7 101. Kf7 Bc5 102. e7 Kxd7 103. e8=Q+ Kd6
104. Kg8 Kd5 105. Qa8+ Kd4 106. Qc8 Bb4 107. Qd7+ Kc3 108. Qe8 Kd4
109. Qf7 Kc3 110. Qg7 Bc5 111. Qh8 Kc4 112. Qh7 Kc3 113. Qg7 Kc4
114. Qh8 Bd4 115. Qh7 Bc5 116. Qg6 Kc3 117. Qf7 Bd6 118. Qe8 Bc5
119. Kg7 Kd3 120. Kg6 Kd4 121. Kf5 Bb4 122. f7 Kc4 123. Qg8 Kb5
124. Qc8 Bc5 125. Qb8+ Kc4 126. Qa8 Bb4 127. Qa7 Kd3 128. Qa6+ Kd4
129. Qa8 Kc4 130. f8=N Kd3 131. Ng6 Kd4 132. Nf4 Be1 133. Qa7+ Kc4
134. Qb7 Bc3 135. Qc8+ Kb3 136. Qd7 Bh8 137. Qe8 Bc3 138. Qf7+ Kb4
139. Qg8 Be1 140. Qh7 Bd2 141. Qg6 Kc5 142. Qe8 Kd4 143. Qf7 Kc3
144. Qg8 Be1 145. Qh7 Bd2 146. Qg6 Kd4 147. Qe8 Kc5 148. Qa8 Be1
149. Qa7+ Kb5 150. Qb7+ Kc4 151. Qc7+ Kb5 152. Qd7+ Kb4 153. Qe7+ Kb3
154. Qxe1 Kc4 155. Qe8 Kc5 156. Qd8 Kc4 157. Qf8 Kb5 158. Qc8 Kb6
159. Qd8+ Kc5 160. Qe8 Kc4 161. Qf7+ Kb5 162. Qg7 Kc6 163. Qh7 Kb5
164. Qh6 Kc5 165. Qh5 Kd4 166. Qh4 Kc3 167. Qh3+ Kc2 168. Qh8 Kb3
169. Qb8+ Kc4 170. Qa8 Kc3 171. Qb7 Kd4 172. Qa7+ Kc3 173. Qa6 Kb4
174. Qb6+ Kc4 175. Qc7+ Kb4 176. Qd7 Kc5 177. Qe7+ Kc4 178. Qd8 Kc3
179. Qd7 Kc4 180. Qe7 Kc3 181. Ke5 Kc4 182. Kd6 Kd4 183. Qd8 Ke4
184. Qf8 Kd4 185. Qc8 Ke4 186. Qd7 Kxf4 187. Qc8 Ke4 188. Qd8 Kf4
189. Qb8 Ke4 190. Qa7 Kf4 191. Qa6 Ke4 192. Qa5 Kf4 193. Qa4+ Ke3
194. Qa8 Kf4 195. Kd5 Kf5 196. Qa7 Kf6 197. Qb7 Kf5 198. Qb8 Kf6
199. Kd6 Kf5 200. Qa8 Kf4 201. Ke6 Ke3 202. Ke5 Kd3 203. Qa7 Kc3
204. Qb8 Kc4 205. Kd6 Kd4 206. Qe8 Kc3 207. Kc5 Kc2 208. Kc4 Kd2
209. Qe7 Kc1 210. Kc3 Kd1 211. Qe8 Kc1 212. Qe1\mate}. White wins.



\chapter{Endgame techniques}

%% skiminki's computer chess studies
%% Copyright (C) 2019 Sami Kiminki
%%
%% This program is free software: you can redistribute it and/or modify
%% it under the terms of the GNU General Public License as published by
%% the Free Software Foundation, either version 3 of the License, or
%% (at your option) any later version.
%%
%% This program is distributed in the hope that it will be useful,
%% but WITHOUT ANY WARRANTY; without even the implied warranty of
%% MERCHANTABILITY or FITNESS FOR A PARTICULAR PURPOSE.  See the
%% GNU General Public License for more details.
%%
%% You should have received a copy of the GNU General Public License
%% along with this program. If not, see <http://www.gnu.org/licenses/>.

\chessgame{Ethereal 10.88}{Lc0 17.11089}{CCCC 1: Rapid Rumble
  (15|5)\\Stage 1 Round 35}{B90 Najdorf, Byrne (English) attack}%
          {2018-09-10}

Following concepts exemplified:
\begin{enumerate}
\item Prying lines open with pawn-and-piece attacking pawn moves
\item The triangle of interception for pawn and king races
\item Blocking two pawns with a knight
\end{enumerate}

The engines started play from the usual start position.

\mainline{1. e4 c5 2. Nf3 d6 3. d4 cxd4 4. Nxd4 Nf6 5. Nc3 a6}

The popular Najdorf variation of the Sicilian defence.

\mainline{6. Be3}

The Byrne (English) attack.

\mainline{6... e5 7. Nb3 Be6 8. f4}

The first move to diverge from the mainlines. The move is not bad at
all, but makes the game sharper than the most popular move,
\movecomment{8. f3.}

\mainline{8... exf4 9. Bxf4 Nc6 10. a3?!\novelty}

\chessboard

This is a novelty.\footnote{Lichess masters database, accessed Apr
  2019.} The main moves here are \movecomment{10. Qe2} and
\movecomment{10. Qd2,} preparing for long castling. This seemingly
unnecessary move has some potential issues:

\begin{enumerate}
\item Black can play Bxb3 and white has to take with the c-pawn,
  instead of having the additional option to take with the a-pawn.
\item Black has later additional options to open Q-side files by
  pushing the b-pawn.
\end{enumerate}

\mainline{10... Be7 11. Qd2 Nh5 12. Be3 Nf6 13. O-O-O O-O 14. Kb1 b5}
\chessboard[color=red, pgfstyle=straightmove, markmoves={b5-b4}]

Now exercising black's typical Najdorf plan after opposite-side
castling. The threat is now to push \wmove{b4} and force \wmove{axb4}.

\mainline{15. Nd4 Nxd4 16. Bxd4 Rb8 17. Be2 Nd7 18. Rhf1 Bg5 19. Qd3 Qe7 20. Bf2 b4}
\chessboard

Now finally pushing b4. This is a typical attacking move to pry open
files. If the knight moves, \wmove{bxa3} will be played.

\mainline{21. axb4 Rxb4 22. Qxd6 Rfb8 23. b3 Qd8 24. Nd5 Rxe4 25. Rfe1
  Bf5 26. Kb2 Rc8 27. Bd3 Rxe1 28. Bxe1 Bg4 29. Ra1 Nc5 30. Qxd8+
  Rxd8}

\chessboard

White has managed to temper black's attack. The position is still
somewhat complicated, but should be objectively a draw.

\mainline{31. Bc4 Be6 32. Rd1 Rd7 33. Bb4 Ne4 34. Kb1 Bd8 35. Rd4 Nf2
  36. Bc5 Ng4 37. Nb4 Rxd4 38. Bxd4 Bxc4 39. bxc4 Bf6?!}

\chessboard

This is a dubious-looking move, offering extra options for white to
drive the endgame into a desireable direction. The game should still
be a draw. Black had a number of other options, such as
\movecomment{39... Nxh2} taking the pawn, \movecomment{39... f5}
hastening the pawn push, or \movecomment{39... Kf8} bringing the king
to play. But since black played \movecomment{39... Bf6?!}, white has
at least three options to choose from:

\emph{Option 1 --- Deflect the knight from attacking h2, gaining a bit
  of time.}  \variationcurrent{40. Bxf6 Nxf6.}

\emph{Option 2 --- Exchange the bishop for the knight, simplifying the
  position.}  \variationcurrent{40. h3 Bxd4 41. hxg4}. While this
position may look a bit weird at first, white's pawns are all in the
light squares, untouchable by the black bishop, and white has double
passed c-pawns while black's a-passer is weak.

But white decided to go with the third option: \wmove{Bg1.} This move
has multiple purposes: (1) it protects h2; (2) it protects the next
square for the c4-pawn; and (3) it avoids exchanging pieces. The
dark-square bishop alone cannot support the advancement of the c-pawn,
and the knight alone is clumsy. But the bishop and the knight
generally work well together to control squares on the way of the
pawn. This was also the best choice for white.

\mainline{40. Bg1 Ne5 41. c5}

Now is the time for black to play \wmove{Kf8}. The black king needs to
move in time to stop the c5 pawn. Highlighted is the triangle of
interception where the king needs to be after black's move in order to
win the race.

\chessboard[color=blue, pgfstyle=straightmove, markmoves={f8-c5, f8-c8},
  color=red, markmoves={c5-c8},
  color=yellow!50, pgfstyle=color, backareas={c8-c5, d8-d6, e8-e7, f8-f8},
  pgfstyle=straightmove, color=black, markmove={g8-f8}]

However, black did not play \wmove{Kf8} and black pieces alone cannot
block or win the passed pawn. Preventing \wmove{c8=Q} will now cost a
piece.

\mainline{41... a5?? 42. c6 Bd8 43. Bd4 Bc7}

Black at least gets now two pawns for the piece. Note that
\variation{43... axb4 44. Bxe5 Kf8 45. c7 Bxc7 46. Bxc7} would have
been even worse for black.

\mainline{44. Bxe5 Bxe5 45. Na6 Bxh2 46. c7 Bxc7 47. Nxc7 f5}

\chessboard

Here Leela thinks she is somewhat better with black, and this
misevaluation was probably the reason why Leela played the losing
\bmove{41... a5??} move earlier. A beginning player might make a
similar evaluation mistake.

The king-side black pawns surely look intimidating, but this is only
superficial. If one does not calculate and/or spot the pattern to stop
the king-side pawns, it is plausible to think that black has time to
march the king to support the a5-pawn, block white's passed c-pawn,
and overrun the king-side with the pawn wall. If white king intercepts
this plan, surely the white knight and a pawn cannot stop the black
pawn wall?

However, white has an easy plan. The g2 pawn guarantees that at least
one black pawn will be exchanged when the black pawns march
forward. If the knight is in time, it can stop two black pawns with
ease with the L-shaped defensive pattern. In fact, white even has the
time to take a small detour with the knight and take the a5, and then
to blockade the king-side pawns.

When choosing the blockading squares for the knight, it is here
beneficial to block the black pawns as late as possible, because that
would require the black king to move beyond the c4 pawn for support,
and then c4 pawn would have easy time to march forward and queen.

This plan is a forced win for white with the best but not difficult
play.

\mainline{48. Ne6 Kf7 49. Nd8+ Kf6 50. Nb7 a4 51. Nc5 g5 52.
  Nxa4 h5 53. Nc5 h4 54. Kc1 g4 55. Nd3 h3 56. gxh3 gxh3}

As promised, the g2 pawn was able to take one black pawn from the
wall. Two pawns left for black.

\mainline{57. c3 h2 58. Nf2}

\chessboard[pgfstyle=knightmove, color=blue, markmoves={f2-h1, h1-f2}]

Now the knight controls the h1 and f2 squares and the black pawns
cannot advance through those squares without support. However, the
king cannot offer assistance, since the c4-pawn would run.

Further, it is important to note that the knight can jump freely
between the h1 and f2 squares to lose tempi if necessary. Losing (or
gaining) tempi is often important in king-pawn endings, and this
ending is not an exception. With only the white king and pawn versus
black king, this would be an easy draw.

\mainline{58... Ke5 59. Kd2 Kd5 60. Kd3 Ke5 61.
  Ke3 Kd5 62. Nh1}

\chessboard

The first tempo loss, basically asking black to make another
move. However, \movecomment{62. Kd3} and losing the tempo later was
equally good.

Black has here one last attempt to trick a draw, although this variant
was not played: \variationcurrent{62... f4+ 63. Kxf4?? Kc4 64. Kg3
  Kxc3 65. Kxh2} draw. The correct move for white was
\movecomment{63. Kd3!} letting the knight to stop the f-pawn, and not
allowing the king to stray away from the all-important c-pawn.

\mainline{62... Kc4 63. Kd2 Kd5 64. Kd3 f4 65. c4+ Kc5 66. Kc3 Kd6
  67. Kd4 Kd7 68. c5 Kd8 69. c6 Kc8 70. Kc5 Kc7 71. Kb5 f3}

\chessboard

Black does not want to move the king, as this allows white to push
forward with \wmove{Kc6}. But after the pawn moves are exhausted,
black is out of options.

\mainline{72. Kc5 f2 73. Nxf2 Kc8 74. Kb6 Kd8 75. c7+}

\chessboard

Without the knight and black pawn, \wmove{Kc8} would be a draw, since
the only way (\wmove{Kc6}) white can protect the pawn would lead in a
stalemate. But here white can always lose a tempo with a knight move,
and ask black to make another move. So, black simply gives up.

\mainline{75... Kd7 76. Kb7 Ke6 77. c8=Q+ Ke5 78. Qh3 Kf4 79. Qxh2+
  Kf3 80. Qh3+}

Final note. Ethereal follows the quickest distance to zero in
tablebase win positions. \movecomment{80. Kc6} would have delivered
the mate one move earlier.

\mainline{80... Kxf2 81. Kb6 Ke2 82. Kc5 Kf2 83. Kd4 Kg1 84. Kd3 Kf2
  85. Qg4 Kf1 86. Ke3 Ke1 87. Qg1\mate}

White wins.


\end{document}
