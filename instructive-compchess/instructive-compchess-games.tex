%% skiminki's computer chess studies
%% Copyright (C) 2019 Sami Kiminki
%%
%% This program is free software: you can redistribute it and/or modify
%% it under the terms of the GNU General Public License as published by
%% the Free Software Foundation, either version 3 of the License, or
%% (at your option) any later version.
%%
%% This program is distributed in the hope that it will be useful,
%% but WITHOUT ANY WARRANTY; without even the implied warranty of
%% MERCHANTABILITY or FITNESS FOR A PARTICULAR PURPOSE.  See the
%% GNU General Public License for more details.
%%
%% You should have received a copy of the GNU General Public License
%% along with this program. If not, see <http://www.gnu.org/licenses/>.

\documentclass[a4paper,11pt,twocolumn]{report}
\usepackage{times}
\usepackage{textcomp}
\usepackage[protrusion=true,expansion=true]{microtype}
\usepackage[LSBC1,LSBC2,LSBC4,T1]{fontenc}
\usepackage[utf8]{inputenc}
\usepackage{pgf}
\usepackage{xcolor}
\usepackage[skaknew]{chessboard,skak}
\usepackage{xskak}
\usepackage[shortlabels]{enumitem}
\usepackage[USenglish]{isodate}
\usepackage[super]{nth}
\font\sknf=SkakNew-Figurine
\font\sknfbx=SkakNew-FigurineBold
\font\skndia=SkakNew-DiagramT

\title{Instructive Computer Games of Chess:\\What we can learn from the beasts?}
\author{S. Kiminki}

\newcommand{\abbeg}[0]{\emph{e.g.}\@}
\newcommand{\abbie}[0]{\emph{i.e.}\@}
\newcommand{\abbetal}[0]{\emph{et al.}\@}

\newcommand{\chessgame}[5]{%
  \begin{center}%
     \section{\textbf{#1} -- \textbf{#2}}% participants
     \textsf{#3}%                          event

     \small{\textsc{#4}}%                  opening

     \small{\emph{\printdate{#5}}}%        date
  \end{center}%
  \styleC\newgame%
}

\newcommand{\chessgameappendix}[0]{\section*{Appendix}}


% notes: styleC is the nice move pair per line; styleB is the condensed format.

\begin{document}
\maketitle

\chapter{Attacking Chess}

%% skiminki's computer chess studies
%% Copyright (C) 2019 Sami Kiminki
%%
%% This program is free software: you can redistribute it and/or modify
%% it under the terms of the GNU General Public License as published by
%% the Free Software Foundation, either version 3 of the License, or
%% (at your option) any later version.
%%
%% This program is distributed in the hope that it will be useful,
%% but WITHOUT ANY WARRANTY; without even the implied warranty of
%% MERCHANTABILITY or FITNESS FOR A PARTICULAR PURPOSE.  See the
%% GNU General Public License for more details.
%%
%% You should have received a copy of the GNU General Public License
%% along with this program. If not, see <http://www.gnu.org/licenses/>.

\chessgame{Stoofvlees II a11}{Gull 3}%
          {TCEC S16 League 2 (30|5), Round 2.4}%
          {C02 - French: Advance, Euwe variation}%
          {2019-07-29}


\mainline{1. e4 e6 2. d4 d5 3. e5 c5 4. c3 Bd7}

End of book

\mainline{5. Nf3 Nc6 6. Be2 Nge7 7. O-O cxd4!? 8. cxd4}

\chessboard

The more common moves were \movecomment{7... Ng6}
and \movecomment{7... Nf5.} The problem with \movecomment{7... cxd4}
is that it locks down the center and white gets a free attack with
king-side pawn pushes. Maybe it was better not to release the tension
too early and play moves such as \bmove{Qb6,} for instance. Usually,
black wants to keep 2 pawn break options in French
(well-timed \wmove{cxd4} and \wmove{f6}), but now there's only
the \bmove{f6} break available after the unfavorable break.

\mainline{8... Nf5 9. Nc3 a6 10. a3!}

Prophylaxis. This is to prepare \wmove{Bd3.}
Without \wmove{a3,} \wmove{Bd3} would be met with \bmove{Ng4.}
Also, \bmove{Bg4} is now prevented.

\mainline{10... Rc8}

Rooks belong in open files.

\mainline{11. Kh1!\novelty}

Makes room for \wmove{Rg1} or possibly even \wmove{Ng1} in some lines,
preparing for an all-out attack. This is a deep strategical attacking
idea.

\mainline{11... Be7 12. g4 Nh4}

After exchanges, white gets free \wmove{f4} with tempo. However, the
damage was already done and this was the best move for black.

\mainline{13. Nxh4 Bxh4 14. f4 Be7 15. Be3}

Now protecting the pawn on d4 by the dark square bishop,
enabling \wmove{Bd3.}

\chessboard

\mainline{15... O-O?!}

This is not a pretty move to make: castling straight into attack. Both
Leela and Stockfish preferred \movecomment{15... Na5} with a
follow-up \bmove{Nc4}, instead. For example: \movecomment{15... Na5
16. f5 Nc4 17. Bxc4 Rxc4} with play for both sides.

\mainline{16. Bd3}

The point of \wmove{a3} is now made: \bmove{Nb4} is not possible here.

\mainline{16... Na5?}

The time for the thematic \bmove{f6} break was now to undermine
white's attack. But here, a critical tempo is lost and the knight is
where the action is not. The fact that black has castled makes all the
difference for white's attack, and white does not give black time to
play \bmove{Nc4} until the position is beyond repair.

\variation{16... f6! 17.  Qc2 fxe5 18. fxe5 Rxf1+ 19. Rxf1 Nxe5
20. dxe5 d4 21. Bxh7+ Kh8 22. Bxd4 Rc4 23. Be4.} White has to give the
piece back with \movecomment{23... Rxd4}. Black is pawn down but
should be able to hold the draw. Note that trying to keep the extra
piece would lose the queen: \movecomment{23. Be3?? Bc6+ 24. Kg1 Rxg4+
25. Kf2 Rg2+}

\mainline{17. Qc2}

Asking black to weaken the king pawns.

\mainline{17... g6?!}

Pawn to \wmove{h6} was better, as the pawn move \wmove{g6} weakens the
dark squares.

\mainline{18. f5}

\chessboard

\mainline{18... Bg5\onlymove}

Only move. Other moves would have trapped the bishop or lost the queen
to stop the mating threat.

\begin{enumerate}[label=(\alph*)]
\item \variation{18... Qe8 19. f6 Bd8 20. Qf2 Rc6 21. Qh4.}
Now \bmove{Bxf6} is the only move avoiding the immediate mate.
\item \variation{18... Re8 19. fxg6 fxg6 20. Bxg6 hxg6 21. Qxg6+} with
mate in 2.
\item \variation{18... exf5 19. gxf5 Bg5 20. Qe2 Bxe3 21. Qxe3 Qh4
22. Rf4 Qh6 23. fxg6 fxg6 24. Rxf8+ Qxf8 25. Rf1 Qg7 26. e6 Be8
27. Rf3 Kh8 28. Nxd5 Bc6 29. Rf7.} White simply has too many threats
and overruns black in this line.
\item \variation{18... gxf5? 19. gxf5 exf5 20. Rg1+ Kh8 21. Bh6 Rg8 22. Rxg8+ Qxg8
23. Rg1} Any attempt to save the queen will result in quick mate. For
example \movecomment{23... Qd8 24. Qg2} and black can only delay the
mate.
\end{enumerate}

\mainline{19. Bf4}
The point here is to get the rook to f4 with tempo.

\mainline{19... Bxf4}
Black could not have realistically prevented \wmove{Rxf4.} If black
played something else, pawn to \wmove{f6} would have soon followed
cutting the support to the bishop and then \bmove{Bxf4} was forced.

\mainline{20. Rxf4 Qh4}
Black had to play the queen to h4 to parry mating ideas and
crash-through threats. For example:
\variation{20... Nc4 21. Qf2 f6 22. fxe6 Bxe6 23. exf6}

\mainline{21. Raf1!}
Supporting the rook on f4, preparing now the pawn moves f6 and
g5. Black does not have time to make room for the queen on f8, as
would be the case with the premature \movecomment{21. f6?} This would
close black's king-side and slow down the attack. White is still
better and black has to maintain keeping an eye on g7. However, here
black has time to interfere the attack and start some counter-play on
the queen-side, and the most important of all, play the queen to
f8: \movecomment{21... Nc4 22. Rg1 Rc7 23. Qe2 Qg5 24. Rgf1 Rfc8
25. Bxc4 Rxc4 26. Qf2 Qh6 27. g5 Qf8.} The queen on f8 makes all the
difference compared to the game continuation.

\mainline{21... Nc4}
Trying to make room for the queen on f8 would have been a terrible
idea: \variation{21... Rfd8?? 22. fxg6 fxg6 23. Bxg6 hxg6 24. Qxg6+}
and the attack is unstoppable.

\mainline{22. Qe2 b5 23. Nd1 Qh6}
The queen is being forced towards her prison on h8.

\mainline{24. b3 Na5}

Taking the pawn instead with \variation{24... Nxa3} would not have
changed anything, since black will not have time to enjoy the extra
pawns. Now white forces the black queen to the corner where it will
sit as the saddest piece of the board.

\mainline{25. Qf2 Qg7}

Black had to add protection to the f7 square to prevent white crashing
through with \wmove{Rxf7} after \wmove{fxg6.} An interesting try was
to take the pawn on f5 with the e-pawn: \variation{25... exf5 26. gxf5
g5 27. Rf3 f6 28. Qg2 Qg7 29. Ne3 Kh8 30. Ng4 Rc6 31. Rg3 Rg8
32. Qxd5,} but this does not work out for black, either.

\mainline{26. f6 Qh8 27. b4 Nc6 28. g5 Rfd8}

\chessboard

The black rook side-steps to prevent \movecomment{29. Bxg6!} crashing
through. If the bishop is taken, then either \movecomment{30. f7+}
or \movecomment{30. Rh4,} both winning decisively.

White now brings the knight and the h-pawn into attack to force a
break through black's defenses.

\mainline{29. Ne3 Rc7 30. Ng4 Rdc8 31. h4 Kf8 32. Qh2 h5}

Black tries to close the pawn structure, but white of course will not
allow that.

\mainline{33. gxh6 Ke8 34. Qd2 a5 35. bxa5 Ra8 36. h5!}
\chessboard

White is now opening the g-file forcibly. The plan after that is
simple: Rook to g8 to win the queen and/or to support promotion of the
more advanced h-pawn. There is no feasible way to stop white's plan.

\mainline{36... g5 37. h7! gxf4 38. Nh6}

Now the Rg1-g8 plan can only be stopped by sacrificing material. Nxf7
attacking the promotion square could also be played given an
opportunity.

\mainline{38... Nxe5 39. dxe5}
This is now forced mate in 15 with best play. Both sides made some
small inaccuracies here which would not change the outcome.

\mainline{39... Bc6 40. Rg1 Qxh7}

The most resilient defense: \variation{40... d4+ 41. Kh2 Kd7 42. Nxf7
Qf8 43.  Rg7 Kc8 44. Nd6+ Kd8 45. Qc1 Rb8 46. Qc5 Bd7 47. Nb7+ Kc8
48. Qxf8+ Kxb7 49.  Qd6 Ka7 50. Qxc7+ Rb7 51. Qc5+ Ka6 52. Rxd7 Kxa5
53. Rxb7 Ka6 54. Qb6#}

\mainline{41. Bxh7 Kd8 42. Kh2 Kc8 43. Nxf7 Rxf7 44. Bg8 Rb8}
Desperate moves, as black was already facing a forced mate.

\mainline{45. Bxf7 Kb7 46. Bxe6 Rh8 47. Rg7+ Ka8 48.  Qd4 Bb7 49. Qb6
Rxh5+ 50. Kg2 Rg5+ 51. Rxg5 d4+ 52. Kf2 d3 53. Rg8+ Bc8 54.  Rxc8#}
White wins.


%% skiminki's computer chess studies
%% Copyright (C) 2019 Sami Kiminki
%%
%% This program is free software: you can redistribute it and/or modify
%% it under the terms of the GNU General Public License as published by
%% the Free Software Foundation, either version 3 of the License, or
%% (at your option) any later version.
%%
%% This program is distributed in the hope that it will be useful,
%% but WITHOUT ANY WARRANTY; without even the implied warranty of
%% MERCHANTABILITY or FITNESS FOR A PARTICULAR PURPOSE.  See the
%% GNU General Public License for more details.
%%
%% You should have received a copy of the GNU General Public License
%% along with this program. If not, see <http://www.gnu.org/licenses/>.

\chessgame{Stoofvlees II a11}{chess22k 1.13}%
          {TCEC S16 League 2, Round 23.1}%
          {D05 - Colle System: Rubinstein Opening}%
          {2019-08-06}

Chess against a strong opponent can be a brutal endeavour, as black
found out in this game the hard way. Black started with some small
inaccuracies in the opening, which culminated in a strategic blunder
in the middle game. An engine as strong as Stoofvlees would not let
such opportunity pass.

Main points:
\begin{enumerate}
\item Opening discussion
\item Positional defensive weaknesses
\item Attacking tactics
\end{enumerate}

\mainline{1. d4 Nf6 2. Nf3 e6 3. e3 c5 4. Bd3 d5}

End of the opening book.

\mainline{5. b3 cxd4?!}

Black is perhaps releasing the tension a bit too eagerly. Now the dark
square bishop does not get an access to the c5 square, as would happen
if white could be persuaded to play \wmove{dxc5.} And indeed, the
mainline \movecomment{5... Nc6} scores significantly better than the
move played.\footnote{\movecomment{5... Nc6} 29\%-41\%-30\%/612
games; \movecomment{5... cxd4} 35\%-45\%-20\%/49 games; Lichess
masters database accessed on \printdate{2019-08-08}.}

\mainline{6. exd4 Nc6!?}

However, now after the exchange, perhaps better was to ask white to
make a slightly awkward pawn move with \variation{6... Bb4+ 7. c3}. The
pawn on c3 would block the bishop's vision, at least temporarily.

\mainline{7. Bb2 b6?!}

Much more common plays were \wmove{Bd6}, \wmove{Be7}, or \wmove{Bb4+.}
Pawn to \wmove{b6} superficially helps the light square bishop
development allowing \wmove{Bg7}. But it's not easy to see how black
could break the center to liberate the bishop. Perhaps better idea was
to play \wmove{Bd7}, instead, and \wmove{Bb5} later given a
chance. Further, as black still had the c-pawn anymore, the move would
have been sensible to support \wmove{c5}. But this is not the case
here.

\mainline{8. Nbd2\novelty}

\chessboard[
        pgfstyle=straightmove,
        color=blue, markmoves={c2-c4},
        pgfstyle=knightmove,
        color=blue,  markmoves={d2-c4},
        color=green, markmoves={d2-f3, f3-e5},
        color=red,   markmoves={d2-e4},
        color=red!30, pgfstyle=color, colorbackfields={e4}]

This is a flexible move:
\begin{enumerate}
\item The knight is ready to
hop in to f3 after the Nf3 knight moves to, say, e5 (green); and
\item Extra control is added on \wmove{d4} to discourage black's
potential \wmove{Ne4}, \wmove{f5} ideas (red); and
\item Support for the potential c4 push is added (blue).
\end{enumerate}

\mainline{8... Bb7 9. a3}

Restricting black's play by preventing \wmove{Ng4} and \wmove{Bg4.}
Also, preparing to meet black's a/b pawn pushes by adding the option
of fixing the queen-side pawns.

\mainline{9... Bd6 10. Ne5 O-O 11. Qe2 Nd7?!}

While seemingly logical in asking white what to do with \wmove{Ne5},
the problem here is that an important king-side defender is
displaced. White has pieces and pawns ready for an attack against the
king.

\mainline{12. O-O}

Castling supports the f-pawn push. That was probably the best
attacking idea available. Pawn to \wmove{f5} would begin to question
black's already weakened control of the center.

\mainline{12... Ncxe5 13. dxe5 Be7}

\wmove{Bc5} would have been met with immediate \wmove{b4} gaining a
tempo for white (\variation{13... Bc5 14. b4 Be7.}) \wmove{Bc7} would
have blocked the rook's access to c-file. With all likelyhood, the
best move was played.

\mainline{14. Nf3}

Now aiming for \wmove{Nd4}, blocking the d-pawn, and thus,
keeping \wmove{Bb7} off the play. Note that white's counterpart bishop
is significantly better, since it
can reroute itself via c1 if necessary, and it already supports the f-pawn
push nicely by protecting e5.

\mainline{14... Nc5 15. Nd4 Rc8}

Rooks belong to open files.

\mainline{16. f4 g6?!}

\chessboard[
        color=red!30, colorbackfields={f6,h6,g7}]

While not outright losing, it can be questioned whether black had to
weaken the king-side pawn structure, and particularly the dark
squares. In fact, this move can be considered as a strategic blunder,
as it opens new avenues for white's attack.

If black was afraid of \wmove{Bxh7+} or \wmove{Qh5}, the moves here to
play were either \wmove{Nxd3} or \wmove{h6.}

Stoofvlees was expecting \wmove{Rc7,} which adds defenses for
the \nth{7} rank and prepares for counterplay in the c-file. But
before playing \bmove{Rc7}, the move that had to be calculated
carefully was \wmove{Bxh7+.} But it turns out that the bishop
sacrifice would not have been mating, as black is just in time to
organize defenses. Variation \variation{16... Rc7 17. Bxh7+ Kxh7
18. Qh5+ Kg8 19. Rf3 Qe8 20. Rh3 f5 21. Qh8+ Kf7 22. Qh5+ Kg8 23. Qh7+
Kf7} would end peacefully.

\mainline{17. Rf3 a6}

White wastes no time, but the same cannot be said about black. Black
either needed to start diluting and preparing for the attack by,
e.g., \wmove{Nxd3}, \wmove{Kh8}, and \wmove{Rg8} to avoid pins and
putting counter-pressure on the g-file; and/or start preparing active
counterplay with \wmove{Rc7} intending to make something happen on the
semi-open c-file. \wmove{Rc7} would also add a defender on the nth{7}
rank provided \wmove{Be7} moves somewhere with \wmove{f5.}

\mainline{18. h4}

Stoofvlees offers a pawn in hope for opening files for attack. With
this Greek gift that should not be accepted, Stoofvlees's evaluation
jumped a bit. However, Stockfish suggests that black is still holding
with \wmove{Nxd3.}

\mainline{18... Bxh4?}

\chessboard

Black was too greedy. White has now the h-file available with strong
attack potential, and an engine as strong as Stoofvlees will not miss
the opportunity.

\mainline{19. Rh3! Be7}

Pawn to \wmove{g4} is coming, so the bishop needed to run either now,
or a concrete plan was needed to meet the follow-up move \wmove{Qh2.}
But regardless, there is no more defense anymore for black if white
plays precisely.

An example line: \variation{19... Rc7 20. Qg4 Nxd3 21. cxd3 Be7
22. Rh6 Kg7 23. f5! exf5 24. Qh3 Rh8 25. e6! f6 26. Rf1 Kg8 27. Nxf5!
Bc5+ 28. Nd4 Rg7 29. b4 Be7 30. Ne2 a5 31. Qe3 Bc8 32. Bd4 axb4
33. axb4 Qd6 34. Bxf6 Qxe6 35. Qd4 Bf8 36. Rh4 h5 37. Nf4 Qd6 38. Be5
Qxb4 39. Qxd5+ Rf7 40. Bxh8 Qc5+ 41. Bd4 Qxd5 42. Nxd5} and white wins
easily with an extra rook.

\mainline{20. g4 Nxd3 21. cxd3 b5 22. f5 Bg5}

The position is already desperate. While allowing the bishop to get
trapped with \wmove{f6,} the alternatives were not much better. For
example:

\begin{enumerate}[label=(\alph*)]
\item \variation{22... exf5 23. Qh2 h5 24. gxf5 Bg5 25. Kh1 Qc7
26. Qg2 Qe7 27. Rg1.} The attempt to save the bishop will end quickly:
\movecomment{27... f6 28. exf6 Bxf6 29. Qxg6+} with forced mate in 13.
\item \variation{22... Bc5 23. f6 Re8 24. Qh2 h5
25. Qf4 Qb6 26. Rxh5} with mate in 10.
\end{enumerate}

\mainline{23. Qh2 h5 24. f6 Bh6}

\chessboard

\mainline{25. Rxh5!}

Nice finishing touch. With the best play, mate would follow in 12 more
moves. The final idea here is to force the queen to g7 with an
unstoppable mate. The bishop can be dealt with g5.

\mainline{25... gxh5}

\variation{25... Be3+} would have postponed the inevitable by one
move. The most resilient continuation was \movecomment{26. Kf1 gxh5
27. Qxh5 Qa5 28. b4 Qc7 29. Re1 Bf4 30. g5 Bxe5 31. Ne2 d4 32. Qh6
Bg2+ 33. Kf2 Bg3+ 34. Kxg2 Qc6+ 35. Kh3 Qg2+ 36. Kxg2 Bxe1 37. Qg7#}


\mainline{26. Qxh5 Bf4 27. Rf1}

Faster was to immediately cut the bishop from defending
with \movecomment{27. g5} with mate in 6.

\mainline{27... Qa5 28. b4 Qa4 29. g5 Rc1
30. Bxc1 Bh2+ 31. Qxh2 Qxb4 32. axb4 Bc6 33. Qh6 Ba8 34. Qg7#}

White wins.


\chapter{Strategic Advantage}

%% skiminki's computer chess studies
%% Copyright (C) 2019 Sami Kiminki
%%
%% This program is free software: you can redistribute it and/or modify
%% it under the terms of the GNU General Public License as published by
%% the Free Software Foundation, either version 3 of the License, or
%% (at your option) any later version.
%%
%% This program is distributed in the hope that it will be useful,
%% but WITHOUT ANY WARRANTY; without even the implied warranty of
%% MERCHANTABILITY or FITNESS FOR A PARTICULAR PURPOSE.  See the
%% GNU General Public License for more details.
%%
%% You should have received a copy of the GNU General Public License
%% along with this program. If not, see <http://www.gnu.org/licenses/>.

\chessgame{LCZero v0.21.1-nT40.T6.532}{Bluefish Dev}%
          {TCEC S15 Bonus: Bluefish vs Leela Jhorthos, Game 2}%
          {E05 Catalan, Open, Classical line}%
          {2019-04-28}

This game is a remarkable example of taking and keeping strategic
advantage. After the opening phase, it feels that white dictated the
direction and black reacted, with black never having a say for where
the game would be heading to.

Following concepts exemplified:
\begin{enumerate}
\item Restricting opponent pieces for strategic advantage
\item Shifting focus from one side of the board to another, to take
advantage of better piece mobility
\end{enumerate}

The game started from the regular starting position.

\mainline{1. d4 Nf6 2. c4 e6 3. g3}

The Catalan opening seems to be a favoured Queen's Gambit opening for
the current NN engines.

\mainline{3... d5 4. Bg2 Be7 5. Nf3 O-O 6. O-O dxc4
  7. Qc2 b5?!}

\chessboard

Typical play here is \movecomment{7... a6,} instead. Trying to hold on
to c4 allows white to get a significant amount for activity for the
pawn.

\mainline{8. a4 b4 9. Nfd2}

While \movecomment{9. Nbd2} looks perhaps a bit more natural and is
slightly more popular in the Lichess GM
database, \movecomment{9. Nfd2} has better statistics for white. The
merit of the move in the game is that it unblocks the bishop's vision,
adding pressure towards the a8 square.

Black has here three options to protect the
rook. \movecomment{9... Nd5} gives up the pawn back and allows black
to develop the queenside. \movecomment{9... b3 10. Qxc4 Ba6 11. Qxb3}
and either \movecomment{11... c6} and black will get one pawn back, or
\movecomment{11... Nd5} for trading one pawn for better development. A
third and the most popular option was played in the game, postponing
the resolution of the queenside development.

\mainline{9... c6 10. Nxc4 Qxd4 11. Rd1 Qc5 12. Be3 Qh5 13. Nbd2 Nd5!?\novelty}

\chessboard

\movecomment{13... Ng4} was the most popular move. This move has the
merit over the played move that it effectively forces
\movecomment{14. Nf3} or \movecomment{14. Nf1} in order to stop the
mate threat, unless white decides to weaken the king-side pawn
structure by \wmove{h3} or \wmove{h4.} This deflects the knight from
the d2 square.

\movecomment{13... Nd5} would make more sense if the purpose was to
block the g2-bishop eyeing towards the black's queen-side
corner. However, the intention was to trade the knight with the
bishop.

\mainline{14. Nb3 Nxe3?! 15. Nxe3}

\chessboard[pgfstyle=straightmove,
  color=green, markmoves={d1-d8},
  color=red, markmoves={c2-c6, g2-c6},
  pgfstyle=knightmove, markmoves={b3-a5, a5-c6}]

Now the small subtlety of playing \movecomment{13... Nd5} instead of
\movecomment{13... Ng5} becomes obvious: black is going to have some
serious questions to answer on developing the queen side, as the c6
pawn is starting to become a liability. Further, white has gained the
d-file, thwarting development ideas such as \wmove{Bd7} with
\wmove{Na6} for now. So, black goes with another typical development
idea in Catalan: push the a-pawn, play \wmove{Ra7}, and then untangle
with moves such as \wmove{Bg7} and \wmove{c5}.

\mainline{15... a6}

Now move \wmove{Ra7} is enabled. The move \movecomment{15... a6} over
the move \movecomment{15... a5} has the following two benefits: (1)
The a-pawn will not become a target for a later
\wmove{Nxa5,} and (2) the pawn controls the b5 square after \wmove{c5.}

\mainline{16. Nc4 Ra7 17. Rac1 c5 18. Nba5}

\chessboard[color=red!30, pgfstyle=color,
  colorbackfields={a8,b7,c6,d6,b6},
  color=yellow!50, colorbackfields={d7},
  color=blue, pgfstyle=straightmove, markmoves={c4-b6, c4-d6},
  color=red, pgfstyle=straightmove, markmoves={c2-c5}]

The b4-pawn is now finally protected, but black's problems are still
far from over. The black queen-side is a minefield due to white's
control, and the d7-square is a bottleneck for black
development. Further, should the e7-bishop move, white is ready to
jump the c4-knight exposing the threat to take the c5-pawn.

Therefore, it is no wonder that it is already getting difficult to
find any useful moves for black:

\begin{enumerate}[label=(\alph*)]
\item \movecomment{18... Rc7} does not help development, since the c6
square is already attacked twice by white. Similarly, a natural
move \movecomment{18... Bg7} cannot be played.
\item \movecomment{18... Qg6} only asks white to make another
developing move such as \movecomment{19. Rd3/e4/Be4}.
\item \movecomment{18... Rd1} allows white to practically
force \movecomment{19. Rxd8+ Bxd8 20. Rd1 Bxa5 21. Nxa5 Qg5 22. Rd6
Rc7 23. Qd1 Nd7 24. Rc6 Rxc6 25. Nxc6.} The
\movecomment{26... c4} move here would be met with \movecomment{27. Qd6}
threatening to either win a piece by \wmove{Ne7+} or a pawn
by \wmove{Qxb4} while still keeping black's queenside development
cumbersome.
\item \movecomment{18... Bd7} with a possible
continuation \movecomment{19. b3 Rc7 20. Qd3 Rd8 21. Qe3 Be8 22. Qf4
Rcd7 23. Rxd7 Bxd7 24. e3} threatening \wmove{Qc7}, for instance.
\end{enumerate}

Black simply does not seem to be able to find any useful counterplay,
so black decided to play a semi-waiting queen move, reinforcing the d8
square.

\mainline{18... Qg5 19. h4! Qf6}
\chessboard

As black is unable to make progress, white is now starting to improve
her position. The move \wmove{h4} not only restricts the queen, but it
also signals white's intention to shift the play in the
king-side. Often such ideas are useful when one side has better access
to squares.

\mainline{20. Qe4 Rc7 21. Rd3 g5}

Black was soon forced to do something. If black would continue to play
waiting moves, then white would soon break black's position. An
example line \variation{21... Kh8 22. b3 Kg8 23. Rcd1 Kh8 24. Rf3 Qh6
25. Qe5} with unparriable threats. For example: \movecomment{25...
Rd7 26. Rfd3 Rxd3 27. Rxd3 Qc1+ 28. Bf1 Nd7 29. Qc7 Bf6 30. Nc6 a5
31. Na7 Ba6 32. Rxd7} with no hope for black.

\mainline{22. h5! Qg7 23. h6!!}
\chessboard

With the h-pawn moves, white continues to strangle black's position,
and begins to threaten a king-side attack. Note that the h6-pawn
cannot be taken as black queen must be ready to stop \wmove{Qe5}, as
that would win a piece due to the skewer (\movecomment{23... Qxh6??
24. Qe5 Rd7 25. Qxb8).}

\mainline{23... Qf6 24. b3}

Solidifying move by white and preventing potential tactics
by \bmove{b3}. Note that often with such moves, white needs to pay
attention not to trap pieces, as b3 was the natural escape square for
the a5-knight. However, here white could simply move the c4-knight
somewhere to provide another escape square, should the need arise.

\mainline{24... Bd7 25. Qe3 g4 26. Rcd1 Be8 27. Kh2}
\chessboard

The king move is a subtle move in this complex position. Black is now
almost in a Zugswang where it would be preferable not to make a move
at all. The king move is also preparatory to avoid tempo loss
by \wmove{Kh8/Rg8+} in some variations after both the g-pawns have
moved to other files, as well as to prepare Rh1 to protect the h6-pawn
with possible rook lift ideas. Finally, \wmove{Kh2}
prepares \wmove{Kg3} to attack the g4-pawn should the opportunity
arise. Let us review some of black's choices:

\begin{enumerate}[label=(\alph*)]
\item \movecomment{27... Qf5} would essentially transpose to the game
continuation after \movecomment{28. Qf4}
\item \movecomment{27... Kh8 28. Ne5 Qf5 29. Qf4 Qxf4 30. gxf4} and
white maintains the advantage with a continuation such
as \movecomment{30... f6 31. Nec4 Bg6 32. R3d2 Rg8 33. Kg3 Bf8 34. e3
Bxh6 35. Rd8 Re7 36. Nb7 Rxd8 37. Rxd8+ Re8 38. Rxe8+ Bxe8 39. Nxc5.}
Note that a move \movecomment{29... Qh5+} would not save the day,
since \movecomment{30. Kg1 Qf5 31. Qxf5 exf5} would just leave white
with a better pawn structure with otherwise similar prospects.
\item \movecomment{27... Bd7 28. Ne5 Be8 29. Nxg4} and white simply
wins a free pawn.
\item \movecomment{27... Ra7} followed by, \emph{e.g.}, \movecomment{28. Qe4 Qxh6+
29. Kg1 Rc7 30. Qxg4+ Qg5 31. Qxg5+ Bxg5 32. Rd6} and white gets a
strong foothold into black's territory, renewing the question on the
queen-side weaknesses. Also, the exchange sacrifice
after \movecomment{28. Rd6 Bxd6 29. Rxd6 Rd8 30. Rxd8 Qxd8 31. Qe5 f6
32. Qxe6+} looks interesting with good compensation, although probably
not necessary to take a risk.
\item \movecomment{27... Ra7 28. Nb7 Qg5 29. Qxg5+ Bxg5 30. Nxc5 Rc7
31. Ne4 Bxh6 32. Nf6+ Kg7 33. Nxe8+ Rxe8 34. Rd4} does not look very
attractive, either.
\end{enumerate}

None of the options look particularly good, so black went with the
straightforward queen exchange.
\mainline{27... Qg5 28. Qf4!}

White does not exchange the queen immediately, but forces better
terms. The exchange on f4 square prevents black to get into the game,
which \variation{28. Qxg5+ Bxg5 29. Rd6 Bxh6 30. Rb6 Bd7 31. Ne5 Bg7}
would have allowed.

\mainline{28... Qxf4}

Black is forced to exchange the queens on white's terms. Note that
curiously, black cannot make a waiting move \movecomment{28... Rc8}
asking again to exchange the queens on the g5 square. The sequence
after \movecomment{29. Qxg5} no longer works because the
move \wmove{Bd7} is no longer possible because the rook had
moved. Also, the intermediate queen check is no good either, because
after \variation{28... Qh5+ 29. Kg1 e5 30. Nxe5 Bg5 31. Qf5 Qxh6
32. Nxg4 Qg6 33. Qe5} black simply drops a piece under white's
threats.

\mainline{29. gxf4}
\chessboard[color=red!30, pgfstyle=color,
  colorbackfields={e5,g5},
  color=black, pgfstyle=straightmove, markmoves={f4-e5, f4-g5}]

The pawn now controls two important squares: e5 provides an anchor
point for a knight, and g5 protects indirectly the h6-pawn by
preventing \wmove{Bg5}.

\mainline{29... Nd7 30. Nc6 Bh4 31. Kg1}
The threat was more important than the execution. Now its time to get
back and protect the f2-pawn.

\mainline{31... Nf6 32. N6e5 Nh5 33. e3 g3}

Black is trying to create counterplay by trying to open the g-file and
repositioning minor pieces in the h-file. However, white has a simple
answer:

\mainline{34. f3!}
\chessboard[color=yellow!50, pgfstyle=color, colorbackfields={h4,h5}]

White simply makes black's own pawn a shield for the potential attack
on the g-file, claiming that the pawn is rather weak. Now the black
minor pieces on the h-file are irreparably mispositioned, and white
can concentrate again on the queen-side. This time black has no answer
to prevent white's penetration with the rooks.

\mainline{34... Be7 35. Nd7! Bxd7 36. Rxd7 Rfc8 37. Nb6! Rxd7 38. Rxd7
Re8}

\chessboard[
    pgfstyle=straightmove, color=green, markmoves={g2-f1, f1-c4, f4-f5},
    color=blue, markmoves={g1-g2, g2-g3},
    color=yellow, markmoves={d7-a7, a7-a5},
    color=red, markmoves={d6-e8},
    pgfstyle=knightmove, color=green, markmoves={b6-c4, c4-d6, d6-f7},
]

White has now multiple pieces for the plan to finish the game:
\begin{enumerate}[label=(\alph*)]
\item The rook can claim the a5-pawn, creating a passed pawn for
white. The cost is some time. (Yellow arrows)
\item The king can march to g2 after the bishop has moved. Then, if
the black knight moves, the g3 pawn can be taken. (Blue arrows)
\item The knight can move to d6 driving \wmove{Re8} away, allowing the
white rook to add additional pressure on f7. The bishop can reposition
to c4, and the pawn can move to f5 adding even more pressure to the
e6/f7 squares. This should allow white to create passed connected
passers on e/f-files. (Green arrows)
\end{enumerate}
Meanwhile, white's h6-pawn has now become weak and cannot be
protected. Some precision is still required.

\mainline{39. Bf1 Nf6 40. Rb7 Nd5 41. Nc4 Bf8 42. Kg2 Bxh6 43. Kxg3}
The king has now claimed the g3 pawn. Note that \movecomment{40. Ra7}
would probably have been slightly more precise, although it does not
matter.

\mainline{43... Bf8 44. e4 Ne7 45. Nd6 Ra8 46. Rd7}

A waiting move to ask black to move again, and to prevent the black to
move \bmove{Rd8} with tempo on the knight.

\mainline{46... a5 47. Bc4}

\chessboard[pgfstyle=straightmove, color=green, markmoves={d7-f7, c4-f7},
            markmoves={d6-f7, f4-f5}]

Now focusing on the f7 pawn with f4 pawn waiting to be moved.

\mainline{47... Nc8 48. Nxc8 Rxc8}

The pawn move f5 is now temporarily discouraged due to
tactics: \variationcurrent{49. f5 Rc7 50. Rxc7 Bd6+ 51. f4 Bxc7.}
However, this would not be a disaster, since
after \movecomment{52. fxe6 fxe6 53. Bxe6+ Kg7 54. e5} the
opposite-colored bishop ending would still be winning for
white. White's connected passed pawns and white king's access to
black's a-b-c pawns are strong enough to win the game. A bishop cannot
usually stop attack on both sides of the board, since one side can
deflect the bishop while the other side decides the game.

However, white did not go into such complications, and simply switched
to taking the a5-pawn first and then resolving the pin by moving the
e4-pawn before pushing the f-pawn.

\mainline{49. Ra7 Bd6 50. Rxa5 Kf8 51. e5}
\chessboard

Black has no good squares for the bishop. \wmove{Bb8} would lose
quickly to \wmove{Ra8} pinning, and then white pushing the a-pawn.

\wmove{51... Be7} would postpone the game by some moves with the
following possible continuation: \movecomment{52. Be7 52. Ra7 Ke8
53. Ba6! Rb8 54. Bb5+ Kf8 55. Bc4.} Note the
move \movecomment{53. Ba6!}. It drives the rook away from the
c8-square, forcing \bmove{Kf8} after the check. The king cannot come
to d8-square after \bmove{Rd8} since it's already occupied, or
after \bmove{Rb8} since then white would play \wmove{Rb7+} with reveal
check, winning the rook.

So, black sacrificed the c5-pawn in order to be able to block the
a-pawn with a bishop.

\mainline{51... Bc7 52. Rxc5 Ke7 53. a5 Kd8 54. a6 Bb6 55. Rb5 Ba7
56. f5}
\chessboard

White is now finally going after the f7-pawn. In opposite-color bishop
endings, it is important to have two passed pawns with some distance
to make it impossible for the bishop to stop both.

\mainline{56... Rb8 57. Ra5}
White could have also exchanged the rooks. The rest is simple
technique.

\mainline{57... exf5 58. Bxf7 Bb6 59. Ra2}
The final pitfall was the natural-looking \wmove{Rb5}. This would have
blundered the win away to simple tactics: \variation{59. Rb5 Bf2+
60. Kxf2 Rxb5} with a disappointing draw. But now the rest is simple
technique.

\mainline{59... Rc8 60. Kf4 h5 61. Kxf5 Rc7 62. Bxh5 Rh7 63. Bg4 Rc7
64. Ra4 Ra7 65. Rxb4 Rxa6}

\chessboard

A quick way to force mate would have been: \variationcurrent{66. Ra4
Ra5 67. Rxa5 Bxa5 68. Ke6 Ke8 69. f4 Bb4 70. f5 Kf8 71. f6 Be1 72. Bh5
Bd2 73. Kd5 Bc3 74. b4 Bxb4 75. e6 Ba3 76. Kc6 Be7 77. fxe7+ Kxe7
78. Bf7 Kf6 79. Kd6 Kf5 80. e7 Ke4 81. e8=Q+ Kd3 82. Qe1 Kc2 83. Qe3
Kb2 84. Qd2+ Kb1 85. Be6 Ka1 86. Qc1\mate}

However, in this game, Leela did not have the distance-to-zero (DTZ)
tablebase files available and was on her own after reaching the
winning position as per the win-draw-loss (WDL) files. As is
characteristic to the current neural network (NN) engines, the endgame
was not concluded quickly. The NN engines are simply interested in
winning the game, not winning the game quickly.

\styleB
\mainline{
66. f4 Bc5 67. Rc4 Bf2 68. Ke4 Rg6 69. Kf3 Bh4 70. b4 Ra6 71. Bf5 Be1
72. b5 Ra3+ 73. Kg4 Rg3+ 74. Kh5 Ba5 75. Ra4 Bb6 76. Bg4 Ke8 77. Kg5
Bd8+ 78. Kf5 Bb6 79. Ra6 Bf2 80. Bh5+ Ke7 81. Re6+ Kd7 82. Be8+ Kc7
83. Rc6+ Kb8 84. Rc4 Ka7 85. e6 Rg7 86. Ke5 Bh4 87. f5 Be7 88. Bg6 Bf8
89. Rc6 Rb7 90. Kf6 Be7+ 91. Kg7 Ba3+ 92. Kh6 Rxb5 93. f6 Bf8+ 94. Kh7
Rb7+ 95. Bf7 Ba3 96. Rc2 Kb6 97. Kg6 Kb5 98. Be8+ Kb6 99. Bd7 Rc7
100. Rxc7 Kxc7 101. Kf7 Bc5 102. e7 Kxd7 103. e8=Q+ Kd6
104. Kg8 Kd5 105. Qa8+ Kd4 106. Qc8 Bb4 107. Qd7+ Kc3 108. Qe8 Kd4
109. Qf7 Kc3 110. Qg7 Bc5 111. Qh8 Kc4 112. Qh7 Kc3 113. Qg7 Kc4
114. Qh8 Bd4 115. Qh7 Bc5 116. Qg6 Kc3 117. Qf7 Bd6 118. Qe8 Bc5
119. Kg7 Kd3 120. Kg6 Kd4 121. Kf5 Bb4 122. f7 Kc4 123. Qg8 Kb5
124. Qc8 Bc5 125. Qb8+ Kc4 126. Qa8 Bb4 127. Qa7 Kd3 128. Qa6+ Kd4
129. Qa8 Kc4 130. f8=N Kd3 131. Ng6 Kd4 132. Nf4 Be1 133. Qa7+ Kc4
134. Qb7 Bc3 135. Qc8+ Kb3 136. Qd7 Bh8 137. Qe8 Bc3 138. Qf7+ Kb4
139. Qg8 Be1 140. Qh7 Bd2 141. Qg6 Kc5 142. Qe8 Kd4 143. Qf7 Kc3
144. Qg8 Be1 145. Qh7 Bd2 146. Qg6 Kd4 147. Qe8 Kc5 148. Qa8 Be1
149. Qa7+ Kb5 150. Qb7+ Kc4 151. Qc7+ Kb5 152. Qd7+ Kb4 153. Qe7+ Kb3
154. Qxe1 Kc4 155. Qe8 Kc5 156. Qd8 Kc4 157. Qf8 Kb5 158. Qc8 Kb6
159. Qd8+ Kc5 160. Qe8 Kc4 161. Qf7+ Kb5 162. Qg7 Kc6 163. Qh7 Kb5
164. Qh6 Kc5 165. Qh5 Kd4 166. Qh4 Kc3 167. Qh3+ Kc2 168. Qh8 Kb3
169. Qb8+ Kc4 170. Qa8 Kc3 171. Qb7 Kd4 172. Qa7+ Kc3 173. Qa6 Kb4
174. Qb6+ Kc4 175. Qc7+ Kb4 176. Qd7 Kc5 177. Qe7+ Kc4 178. Qd8 Kc3
179. Qd7 Kc4 180. Qe7 Kc3 181. Ke5 Kc4 182. Kd6 Kd4 183. Qd8 Ke4
184. Qf8 Kd4 185. Qc8 Ke4 186. Qd7 Kxf4 187. Qc8 Ke4 188. Qd8 Kf4
189. Qb8 Ke4 190. Qa7 Kf4 191. Qa6 Ke4 192. Qa5 Kf4 193. Qa4+ Ke3
194. Qa8 Kf4 195. Kd5 Kf5 196. Qa7 Kf6 197. Qb7 Kf5 198. Qb8 Kf6
199. Kd6 Kf5 200. Qa8 Kf4 201. Ke6 Ke3 202. Ke5 Kd3 203. Qa7 Kc3
204. Qb8 Kc4 205. Kd6 Kd4 206. Qe8 Kc3 207. Kc5 Kc2 208. Kc4 Kd2
209. Qe7 Kc1 210. Kc3 Kd1 211. Qe8 Kc1 212. Qe1\mate}. White wins.



\chapter{Endgame techniques}

%% skiminki's computer chess studies
%% Copyright (C) 2019 Sami Kiminki
%%
%% This program is free software: you can redistribute it and/or modify
%% it under the terms of the GNU General Public License as published by
%% the Free Software Foundation, either version 3 of the License, or
%% (at your option) any later version.
%%
%% This program is distributed in the hope that it will be useful,
%% but WITHOUT ANY WARRANTY; without even the implied warranty of
%% MERCHANTABILITY or FITNESS FOR A PARTICULAR PURPOSE.  See the
%% GNU General Public License for more details.
%%
%% You should have received a copy of the GNU General Public License
%% along with this program. If not, see <http://www.gnu.org/licenses/>.

\chessgame{Ethereal 10.88}{Lc0 17.11089}{CCCC 1: Rapid Rumble
  (15|5)\\Stage 1 Round 35}{B90 Najdorf, Byrne (English) attack}%
          {2018-09-10}

Following concepts exemplified:
\begin{enumerate}
\item Prying lines open with pawn-and-piece attacking pawn moves
\item The triangle of interception for pawn and king races
\item Blocking two pawns with a knight
\end{enumerate}

The engines started play from the usual start position.

\mainline{1. e4 c5 2. Nf3 d6 3. d4 cxd4 4. Nxd4 Nf6 5. Nc3 a6}

The popular Najdorf variation of the Sicilian defence.

\mainline{6. Be3}

The Byrne (English) attack.

\mainline{6... e5 7. Nb3 Be6 8. f4}

The first move to diverge from the mainlines. The move is not bad at
all, but makes the game sharper than the most popular move,
\movecomment{8. f3.}

\mainline{8... exf4 9. Bxf4 Nc6 10. a3?!\novelty}

\chessboard

This is a novelty.\footnote{Lichess masters database, accessed Apr
  2019.} The main moves here are \movecomment{10. Qe2} and
\movecomment{10. Qd2,} preparing for long castling. This seemingly
unnecessary move has some potential issues:

\begin{enumerate}
\item Black can play Bxb3 and white has to take with the c-pawn,
  instead of having the additional option to take with the a-pawn.
\item Black has later additional options to open Q-side files by
  pushing the b-pawn.
\end{enumerate}

\mainline{10... Be7 11. Qd2 Nh5 12. Be3 Nf6 13. O-O-O O-O 14. Kb1 b5}
\chessboard[color=red, pgfstyle=straightmove, markmoves={b5-b4}]

Now exercising black's typical Najdorf plan after opposite-side
castling. The threat is now to push \wmove{b4} and force \wmove{axb4}.

\mainline{15. Nd4 Nxd4 16. Bxd4 Rb8 17. Be2 Nd7 18. Rhf1 Bg5 19. Qd3 Qe7 20. Bf2 b4}
\chessboard

Now finally pushing b4. This is a typical attacking move to pry open
files. If the knight moves, \wmove{bxa3} will be played.

\mainline{21. axb4 Rxb4 22. Qxd6 Rfb8 23. b3 Qd8 24. Nd5 Rxe4 25. Rfe1
  Bf5 26. Kb2 Rc8 27. Bd3 Rxe1 28. Bxe1 Bg4 29. Ra1 Nc5 30. Qxd8+
  Rxd8}

\chessboard

White has managed to temper black's attack. The position is still
somewhat complicated, but should be objectively a draw.

\mainline{31. Bc4 Be6 32. Rd1 Rd7 33. Bb4 Ne4 34. Kb1 Bd8 35. Rd4 Nf2
  36. Bc5 Ng4 37. Nb4 Rxd4 38. Bxd4 Bxc4 39. bxc4 Bf6?!}

\chessboard

This is a dubious-looking move, offering extra options for white to
drive the endgame into a desireable direction. The game should still
be a draw. Black had a number of other options, such as
\movecomment{39... Nxh2} taking the pawn, \movecomment{39... f5}
hastening the pawn push, or \movecomment{39... Kf8} bringing the king
to play. But since black played \movecomment{39... Bf6?!}, white has
at least three options to choose from:

\emph{Option 1 --- Deflect the knight from attacking h2, gaining a bit
  of time.}  \variationcurrent{40. Bxf6 Nxf6.}

\emph{Option 2 --- Exchange the bishop for the knight, simplifying the
  position.}  \variationcurrent{40. h3 Bxd4 41. hxg4}. While this
position may look a bit weird at first, white's pawns are all in the
light squares, untouchable by the black bishop, and white has double
passed c-pawns while black's a-passer is weak.

But white decided to go with the third option: \wmove{Bg1.} This move
has multiple purposes: (1) it protects h2; (2) it protects the next
square for the c4-pawn; and (3) it avoids exchanging pieces. The
dark-square bishop alone cannot support the advancement of the c-pawn,
and the knight alone is clumsy. But the bishop and the knight
generally work well together to control squares on the way of the
pawn. This was also the best choice for white.

\mainline{40. Bg1 Ne5 41. c5}

Now is the time for black to play \wmove{Kf8}. The black king needs to
move in time to stop the c5 pawn. Highlighted is the triangle of
interception where the king needs to be after black's move in order to
win the race.

\chessboard[color=blue, pgfstyle=straightmove, markmoves={f8-c5, f8-c8},
  color=red, markmoves={c5-c8},
  color=yellow!50, pgfstyle=color, backareas={c8-c5, d8-d6, e8-e7, f8-f8},
  pgfstyle=straightmove, color=black, markmove={g8-f8}]

However, black did not play \wmove{Kf8} and black pieces alone cannot
block or win the passed pawn. Preventing \wmove{c8=Q} will now cost a
piece.

\mainline{41... a5?? 42. c6 Bd8 43. Bd4 Bc7}

Black at least gets now two pawns for the piece. Note that
\variation{43... axb4 44. Bxe5 Kf8 45. c7 Bxc7 46. Bxc7} would have
been even worse for black.

\mainline{44. Bxe5 Bxe5 45. Na6 Bxh2 46. c7 Bxc7 47. Nxc7 f5}

\chessboard

Here Leela thinks she is somewhat better with black, and this
misevaluation was probably the reason why Leela played the losing
\bmove{41... a5??} move earlier. A beginning player might make a
similar evaluation mistake.

The king-side black pawns surely look intimidating, but this is only
superficial. If one does not calculate and/or spot the pattern to stop
the king-side pawns, it is plausible to think that black has time to
march the king to support the a5-pawn, block white's passed c-pawn,
and overrun the king-side with the pawn wall. If white king intercepts
this plan, surely the white knight and a pawn cannot stop the black
pawn wall?

However, white has an easy plan. The g2 pawn guarantees that at least
one black pawn will be exchanged when the black pawns march
forward. If the knight is in time, it can stop two black pawns with
ease with the L-shaped defensive pattern. In fact, white even has the
time to take a small detour with the knight and take the a5, and then
to blockade the king-side pawns.

When choosing the blockading squares for the knight, it is here
beneficial to block the black pawns as late as possible, because that
would require the black king to move beyond the c4 pawn for support,
and then c4 pawn would have easy time to march forward and queen.

This plan is a forced win for white with the best but not difficult
play.

\mainline{48. Ne6 Kf7 49. Nd8+ Kf6 50. Nb7 a4 51. Nc5 g5 52.
  Nxa4 h5 53. Nc5 h4 54. Kc1 g4 55. Nd3 h3 56. gxh3 gxh3}

As promised, the g2 pawn was able to take one black pawn from the
wall. Two pawns left for black.

\mainline{57. c3 h2 58. Nf2}

\chessboard[pgfstyle=knightmove, color=blue, markmoves={f2-h1, h1-f2}]

Now the knight controls the h1 and f2 squares and the black pawns
cannot advance through those squares without support. However, the
king cannot offer assistance, since the c4-pawn would run.

Further, it is important to note that the knight can jump freely
between the h1 and f2 squares to lose tempi if necessary. Losing (or
gaining) tempi is often important in king-pawn endings, and this
ending is not an exception. With only the white king and pawn versus
black king, this would be an easy draw.

\mainline{58... Ke5 59. Kd2 Kd5 60. Kd3 Ke5 61.
  Ke3 Kd5 62. Nh1}

\chessboard

The first tempo loss, basically asking black to make another
move. However, \movecomment{62. Kd3} and losing the tempo later was
equally good.

Black has here one last attempt to trick a draw, although this variant
was not played: \variationcurrent{62... f4+ 63. Kxf4?? Kc4 64. Kg3
  Kxc3 65. Kxh2} draw. The correct move for white was
\movecomment{63. Kd3!} letting the knight to stop the f-pawn, and not
allowing the king to stray away from the all-important c-pawn.

\mainline{62... Kc4 63. Kd2 Kd5 64. Kd3 f4 65. c4+ Kc5 66. Kc3 Kd6
  67. Kd4 Kd7 68. c5 Kd8 69. c6 Kc8 70. Kc5 Kc7 71. Kb5 f3}

\chessboard

Black does not want to move the king, as this allows white to push
forward with \wmove{Kc6}. But after the pawn moves are exhausted,
black is out of options.

\mainline{72. Kc5 f2 73. Nxf2 Kc8 74. Kb6 Kd8 75. c7+}

\chessboard

Without the knight and black pawn, \wmove{Kc8} would be a draw, since
the only way (\wmove{Kc6}) white can protect the pawn would lead in a
stalemate. But here white can always lose a tempo with a knight move,
and ask black to make another move. So, black simply gives up.

\mainline{75... Kd7 76. Kb7 Ke6 77. c8=Q+ Ke5 78. Qh3 Kf4 79. Qxh2+
  Kf3 80. Qh3+}

Final note. Ethereal follows the quickest distance to zero in
tablebase win positions. \movecomment{80. Kc6} would have delivered
the mate one move earlier.

\mainline{80... Kxf2 81. Kb6 Ke2 82. Kc5 Kf2 83. Kd4 Kg1 84. Kd3 Kf2
  85. Qg4 Kf1 86. Ke3 Ke1 87. Qg1\mate}

White wins.


%% skiminki's computer chess studies
%% Copyright (C) 2019 Sami Kiminki
%%
%% This program is free software: you can redistribute it and/or modify
%% it under the terms of the GNU General Public License as published by
%% the Free Software Foundation, either version 3 of the License, or
%% (at your option) any later version.
%%
%% This program is distributed in the hope that it will be useful,
%% but WITHOUT ANY WARRANTY; without even the implied warranty of
%% MERCHANTABILITY or FITNESS FOR A PARTICULAR PURPOSE.  See the
%% GNU General Public License for more details.
%%
%% You should have received a copy of the GNU General Public License
%% along with this program. If not, see <http://www.gnu.org/licenses/>.

\chessgame{LCZero v0.21.1-nT40.T8.610}{Stockfish 19050918}%
          {TCEC S15 Superfinal, Game 12}%
          {C05 French, Tarrasch, Closed}%
          {2019-05-12}


Following concepts exemplified:
\begin{enumerate}
\item Importance of the center to shelter the king when the king is
not behind pawns
\item Tactical sequences for positional gains
\item Positional analysis to create a fortress
\item Advanced endgame techniques: deflection and skewer
\end{enumerate}


\mainline{1. e4 e6 2. d4 d5 3. Nd2 Nf6 4. e5 Nfd7 5. f4 c5 6. c3 Nc6
7. Ndf3 Qb6 8. g3 Be7}

The most popular continuation here is \variation{8... cxd4 9. cxd4
Bb4+ 10. Kf2 g5 11. fxg5 Ndxe5 12. Nxe5 Nxe5 13. Kg2 Nc6 14. Nf3 Bf8
15. b3 Bg7 16. Bb2 Bd7}.

\mainline{9. Kf2}

\chessboard[pgfstyle=straightmove,
  color=yellow, markmoves={b6-f2}]

End of the opening book. This is a prophylaxis to avoid \wmove{Bb4}
with a tempo after \movecomment{... cxd4 cxd4}. See the mainline
\movecomment{8... cxd4} for details. But this move is not completely
without drawbacks, and a potential pin has to be considered.

In positions such as this, where the king is not safely behind the
pawns after the usual castling, trying to maintain a strong blockaded
center is often a wise decision. The reason is quite simple: without
the central pawns, black would be able to start attacking the white
king using diagonal through the center, often with double attacks of
forks and mate-threatening tactics. The importance of the central
pawns is exemplified by the queen on b6 eyeing the king.

However, since the center is not yet solidified, white has to be
careful. For instance, the e5-pawn cannot rely on the protection of
the d4 pawn, since after \movecomment{... cxd4 cxd4}, the white
d4-pawn would be pinned. This is not a problem right now, since the
e5-pawn is protected by the f4-pawn and the knight on f3. But, the
potential pin has to be constantly factored in when calculating the
responses to black's attempts to undermine the center with a typical
plan of f6.

A slightly more popular move than \movecomment{9. Kf2} was to
play \movecomment{9. Bh3,} instead. This discourages \bmove{f6} ideas
by exposing the resulting weakness in the e6-pawn.


\mainline{9... a5 10. a4 cxd4!}

This logical move has many upsides. First, it allows installation of a
piece in b4, as c3 will no longer control it. Second, the pawn on d4
becomes a bit weak. Third, white has to spend a tempo in order to move
the queen out of the pin soon. Fourth, the c-file is opened, which
should favor black due to white's king safety issues if black is able
to use by the rooks.

\mainline{11. cxd4 Ndb8}

\chessboard[pgfstyle=knightmove,
  color=green, markmoves={b8-a6,a6-b4},
  color=yellow, markmoves={c6-b4},
  color=red!30, pgfstyle=color,
  colorbackfields={d4}]

With a plan of \wmove{Nb8-a6-b4} installing an extra strong knight on
b4. If white takes, there's another knight ready to step in. Potential
weakness on d4 highlighted.

\mainline{12. Ne2 O-O 13. Kg2}
Now avoiding the pin, simplifying white's play in response
to \bmove{f6}. An interesting alternative would have been to
play \wmove{Bh3} first, allowing white to put some pressure on the
pawn on e6, discouraging \bmove{f6} by positional arguments against
the weakened e6.

\mainline{13... Na6 14. Nc3?!}
Here the validity of the move by white can be questioned. While there
are certainly ideas of playing \wmove{Na2} to challenge the b4 square
and \wmove{Nb5} blocking black queen's access to the b-file, more
importantly, the move also undermines the protection of the d4
pawn. This allows black to execute a better version of the f-pawn
push, forcing white to recapture with the d-pawn instead of the
f-pawn. White could have considered playing \wmove{Rb1}
and \wmove{Be3} first to solidify the d4 pawn, and only then
playing \wmove{Nc3}.

\mainline{14... f6! 15. h4 fxe5 16. dxe5}

\chessboard[pgfstyle=straightmove,
  color=green, markmoves={d5-d4}]

Now the argument against \movecomment{14 Nc3} has been made.
Capturing with the f-pawn would now have allow black sacrificing the
exchange for a knight and pawn as a direct
consequence: \variation{16. fxe5 Rxf3! 17. Qxf3 Nxd4 18. Qd1} with
plenty of compensation. Also, the earlier \movecomment{15. exf6 Bxf6}
would not have been attractive either, as black would be able to put
proper pressure on the pawn on d4. So, white was forced to capture
with the d-pawn, and black has now a scary-looking protected potential
passer on d5, which is now controlling important squares c4 and e4.

\mainline{16... h6 17. Bd3 Nab4}
Black has now finally executed the plan to install an extra strong
knight on b4. Sometimes, it is said that knights protecting each other
are clumsy, because they're in each other's way. And surely, in
endgames, this can be true, especially when the knights are the only
thing protecting each other. However, this is different, since the
knight on c6 serves as a replacement in case the knight on b4 is
captured.

The knight on b4 is ready to support push of the d-pawn up to d3.

\mainline{18. Bb1 Bd7 19. Na2 Rac8 20. Kh3}
\chessboard[pgfstyle=straightmove]

This little move deserves special attention. While \movecomment{20. Kh3} may
look like a prophylaxis, and it is, it also puts the king in the same
diagonal with
\wmove{Bd7.} This makes e6 pawn a bit less weak, since white has to spend an
extra tempo to move the king away from the diagonal before e6 can be
captured without a pin. However, with the king safety being somewhat
questioned, being sheltered by an enemy pawn is probably better than
leaving the king in g2 awaiting for tactics. After the knight moves
away from c6, the king on g2 would be subject to \bmove{d4} with a
follow-up check through the a8-h1 diagonal with tempo gains.

\mainline{20... Be8 21. Nxb4 Nxb4 22. Ra3 Bc5 23. Bd2 Qa6 24. Rc3}

White is not quite in time to defuse black's pressure. If given a free
tempo, say \variationcurrent{24... Kh8 25. Be3 b6 26. Nd4 Bxd4
27. Bxd4}, and white would be able to blockade the d-pawn and perhaps
starting to target the pawn on b6 or preparing \wmove{g4}
and \wmove{f5} with the idea to create a passer on the
e-file. However, tempi are a scarcity in chess.

Instead of trying to keep the tension with \movecomment{24. Rc3,}
maybe it was time to relieve the tension a bit
with \variation{24. Bxb4} and go for a draw.

\mainline{24... b5}
\chessboard[pgfstyle=straightmove]

One thing that always amazes is how the computers so casually allow
pins and leave pieces hanging. But of course, the computers are able
to calculate through tactics. Many humans would understandably start
looking into solidifying moves such as \wmove{Bd7} to
prevent \wmove{Qc1} pinning the bishop on c5. Instead of solidifying
the position and trying to untangle the pins, black complicates the
position to win a pawn.

\mainline{25. Qc1}

Taking the b-pawn would be problematic: \variation{25. axb5?!
Bxb5\withattack} and white would have annoying threats such
as \bmove{Be2} fork to deal with. Pinning the bishop was the better
choice, although black can strenghten the protection of the bishop on
c5 just in time.

\mainline{25... Qb6 26. Nd4 bxa4}

This move was the point of the tactical complications
of \movecomment{24... b5}, changing the queen-side pawn structure to
favor black. White is now left with a weak b-pawn against doubled
a-pawns and a passer on d5.

\mainline{27. Be3 Bd7}

Now black allows a tactical sequence by white winning an exchange. If
black wanted, the next move could have been prevented
by \variation{27... Rc7.} White enters now in a forced sequence.

\mainline{28. Nf5!? Bxe3!}

Note that giving up the exchange early with \variation{28... Rxf5
29. Rxc5 Rxc5} would have given white two pleasant options:
(1) \movecomment{30. Bxc5 Qa6 31. Bxf5 exf5\wupperhand} taking the
exchange with roughly equal pawn structure, as white also has a
passer; or (2) \movecomment{30. Qxc5 Qxc5 31. Bxc5 Rf7 32. Bg6 Bb5
33. Bxf7+ Kxf7\wupperhand} delaying the taking of the exchange a bit,
keeping the pawn structure but simplifying the positions with
exchanges. Both variations are likely winning for white.

\mainline{29. Ne7+ Kf7 30. Nxc8 Rxc8 31. Qxe3 d4}

Forcing sequence ends. Here white could have also untangled from the
fork by \movecomment{32. Qd2,} as \movecomment{32... dxc3??} would
fall for \movecomment{33. Qxd7+ Kf8 34. Qxc8+} with mate soon to
follow.

\mainline{32. Qf2 Qb7 33. Rcc1 d3}

\chessboard[pgfstyle=straightmove]

Now black has finally been able to push the d-pawn to d3. Since the
pawn is protected by the strong knight on b4 and black can enforce the
protection by the light square bishop, white has no good way to
challenge the pawn on d3. Note that the white king cannot come to
assist in the capture either, because after the exchanges, white's
king-side pawns would be subject to be captured by the black king.

We are now going to fast forward to the next critical position.

\mainline{34. h5 Rxc1
35. Rxc1 Qd5 36. Kh4 Bc6 37. g4 Ke8 38. Rf1 Qd8+ 39. Kg3 Qd5 40. Rd1
Qb3 41. Rd2 Qc4 42. Rd1 Qb3 43. Qd2 Be4 44. Re1 Bb7}

\chessboard[pgfstyle=straightmove]

The pawn move \wmove{f5} here is interesting, and it seems a way for
white to force a draw, although the play is not forced by either
side. For example, \variationcurrent{45. f5 exf5 46. gxf5 Qd5 47. Re3}
and black can still hang on to the d3 pawn
with \movecomment{47... Bc8} with the idea of \movecomment{48. Bxd3??
Bxf5!}. But after \movecomment{48. e6 Qxf5 49. Bxd3} it would be black
who has to be careful.

An interesting alternative play for black would
be \movecomment{46... Nc2}. White has no oblication to take on c2, but
it would lead in a nice way to force a draw. \movecomment{47. Bxc2
dxc2+ 48. Re3 Qxb2 49. f6.} Here black could promote to get another
queen, but white would be just in time with \movecomment{49... c1=Q
50. f7+ Kxf7 51. Qd7+ Kg8 52. Qe8+ Kh7 53. Qg6+ Kg8 54. Qe8+}, drawing
with perpetual checks.

\mainline{45. Rd1 Be4
46. Kh4 Qd5 47.  Qc3 Kf7 48. Qc7+ Kg8 49. Qc3 Kh7 50. Kh3 Kg8 51. Kh4
Kh7 52. Kg3 Kg8 53. Qc8+ Kh7 54. Qc1 Qb3 55. Qd2 Kg8 56. g5}

\chessboard[pgfstyle=straightmove]

Taking the pawn on g5 here might be a small
inaccuracy: \variationcurrent{56... hxg5 57. fxg5 g6} to prevent white
from playing g6 himself with backrank mate threats. An example
contituation: \movecomment{58. Rc1 Nc2 59. Kf4 Qb7 60. Bxc2 dxc2
61. Qd8+ Kh7 62. Qf8 gxh5 63. Qh6+ Kg8 64. Qxe6+}, but this should
still be about equal.

\mainline{56... Kf8 57. g6 Ke7 58. Kh3 Ke8 59. Kh4 Qc4 60. Kg3 Bf5
61. Qc3}

\chessboard[pgfstyle=straightmove]

This is a committal move, offering the queen exchange, which black
accepts. While the game has been objectively close to a draw and still
is, only black can press for the win. With queens on the board, there
was always a possibility for some dynamic play.

\mainline{61... Qxc3 62. bxc3}

It is possible that white thought that at this position, the d3-pawn
could be somehow won, possibly by giving the exchange back. If that
was the case, then white king could hold or take the black pawns on
the a-file. However, the d-pawn can never be taken without losing the
game.

\mainline{62... Nc2}

\chessboard[
  color=red!30, pgfstyle=color,
  colorbackfields={e1,e2,e3,e4}]

This knight now becomes a very annoying piece. Together with the pawn
and the bishop, all white king's access squares to the d-pawn are
controlled.

\mainline{63. Kf2}
White now has a threat of \wmove{Bxc2}, and white would be just in
time to stop the black pawns.

\mainline{63... a3!}
Parries the threat of Bxc2. Now black king is just in time to escort
the a-pawns.

\mainline{64. Ba2}
Taking the c2-pawn would have been a huge blunder. \variation{64. Bxc2
dxc2 65. Rc1 Kd7 66. Ke3 Kc6 67. Kd2 Kb5 68. Rxc2 Kc4!} Taking the
rook with Bxc2 would have been a huge blunder, since white has a
winning king-side pawn break. \movecomment{69. Rc1 Kb3 70. Re1 a2} and
there's no way of stopping \wmove{a1Q} other than giving up the
rook.


\mainline{64... Kd7 65. Bb3 Kc7 66. Rg1 Kd7}

\chessboard

This is the final position in the game where the game was still
objectively a draw.

\mainline{67. Kf3?}
Computer analysis suggests that white had at seven moves which would
have maintained the draw. Unfortunately, the move played was none of
them. The problem with \movecomment{67. Kf3?} is that it
allows \movecomment{67... d2} with precise tactics as played, winning
the game.

The key for white maintaining the draw is to set up a dynamic
fortress, preventing black's progress. Let us take a closer look.

Perhaps the easiest way to set up the defenses is the straightforward
Ba4+.

\movecomment{67. Ba4+ Kc7 68. Bb3.} The bishop on b3 and the
pawn on c3 guard the entry squares for the black king, and the bishop
additionally stops the immediate a2 and a4. The rook's job is to
create enough harassment to prevent the black bishop to enter a square
to protect the d1 promotion square, and the d-pawn push. Black would
need two tempi to prepare \wmove{d1=Q}, but will never have enough
time with the best defense.

\movecomment{68... Kc6 69. Rd1.} This is the easiest
plan. Now \wmove{Bg4} is prevented, as d3 would be hanging.

\movecomment{69... Kb5 70. Rb1.} The only
move. Black has to move the king away to prevent Bxc2+ exposure check,
and thus, \wmove{a4} or \wmove{Bg4} here is prevented.

Black has to be careful not to overextend, and thus has to
retreat. The c5-square is off limits for the king here, since that
would allow the white rook to enter the 8th rank. For example:
\movecomment{70... Kc5? 71. Ba2! Kc6. 72. Rb8}, and the white rook would
start picking up the black pawns. There is also no time for black to
play \wmove{d2} in this line, since the rook would simply move to the
d-file to pick up the pawn.

\movecomment{70... Kc6 71. Rd1} to prevent \wmove{Bg4}. Black has no
way to make progress. The final attempt is to play \wmove{a4} after
{Ba2}.

\movecomment{71... Kc7 72. Ba2 a4.}  However, \wmove{Ba2}
and \wmove{c3} will control the entry squares for the king, and
without the king, the d-pawn can never promote successfully.

\mainline{67... d2! 68. Ke2 Ne1!}

The d2-pawn is untouchable, \movecomment{69. Kxd2 Nf3+} and black
picks up the rook.

\mainline{69. Kd1 Nf3}

The pawn on d2 is now protected. However, precise play is still needed
for conversion, but that is no problem for Stockfish.

\mainline{70. Rh1 Bd3}

\chessboard

Black's threat here is to play \wmove{Bb1} and then deflect the white
bishop with \wmove{a4}, and when the bishop is no longer controlling
the a2-square, then play \wmove{a2} and \wmove{a1=Q}. But that has to
be prepared by moving the queen out of \wmove{Bxa4+} check.

\mainline{71. Ba4+ Ke7 72. Bb3 Bb1}
Now the \wmove{a4} deflecting threat is enabled. Now, white is really
out of moves.

\mainline{73. Rh3}

Another try was \variation{73. Rf1 a4 74. Bc4 Bd3! 75. Bxc3 a2,} but
that does not work, either.

\mainline{73... Be4 74. Rh1 Nxe5}

Picks up a pawn. White cannot give the exchange
back: \variationcurrent{75. fxe5 Bxh1 76. Kxd2 a4 77. Ba2 Bf3 78. c4
Bxh5 79. Kc3 Bxg6 80. c5 Be4 81. Bc4 h5} and white cannot stop both a
and h-pawns.


\mainline{75. Rf1 a4 76. Ba2}
\variation{76. fxe5 axb3} with connected unstoppable passers for black.

\mainline{76... Nf3 77. Rh1 Bd3 78. Rh3 Bf5 79. Rh1}

The knight cannot be taken due to skewer. \variation{79. Rxf3 Bg4
80. Ke2 Bxf3+ 81. Kxd2} and black picks up white's h5 and g6 pawns.

\mainline{79... Kd6 80. Bc4 Bb1 81. Rh3}

\chessboard

0-1. Black wins by adjudication.

A possible continuation: \variationcurrent{81... Bd3 82. Ba2 Be2+
83. Kxe2 Ng1+ 84. Kxd2 Nxh3 85. Ke3 e5 86. f5 Nf4 87. f6 Nxh5 88. f7
Ke7 89. Ke4 Nf4 90. Kxe5 Nd3+ 91. Ke4 Nc1 92. Bb1 a2 93. Bxa2 Nxa2
94. Kd3 Nc1+ 95. Kd2 Nb3+ 96. Kc2 h5}. White king cannot stop both a
and h-pawns.


\chessgameappendix{}

\fenboard{8/6p1/4p1Pp/2k1P2P/5P2/8/8/4K3 b - - 0 1}
\chessboard[clearfields={c5,e1}]

When evaluating transitions to endings, it is often useful to analyze
specific pawn structures on the board and whether they're winning or
not given the remaining pieces.

The pawn structure in the figure is winning for white, unless there is
a black piece to stop the queening. The plan for white is to play f5,
and then:
\begin{enumerate}[label=(\alph*)]
\item if black takes, push the e-pawn. Full
variation: \movecomment{1. f5 exf5 2. e6 f4} and the e-pawn runs.
\item if black doesn't take, push the f-pawn again. Full variation:
\movecomment{1. f5} and \movecomment{2. f6 gxf6 3. g7} and the g-pawn
queens on the next move.
\item if black still doesn't take, take the g-pawn with the
f-pawn. Full variation: \movecomment{1. f5} and \movecomment{2. f6}
and \movecomment{3. fxg7} and the g-pawn queens on the next move.
\end{enumerate}

Now, when considering a king-pawn ending with this structure, it is
quite straightforward to determine the area where the black king has
to be in order to stop the queening, provided white king is far
away. Consider that \movecomment{1. f5} has been played.

\fenboard{8/6p1/4p1Pp/2k1PP1P/8/8/8/5K2 b - - 0 68}
\chessboard[clearfields={c5,e1},
  color=blue!30, pgfstyle=color,
  colorbackfields={d4,e4,f4,g4,h4},
  pgfstyle=straightmove,
  color=green, markmoves={b5-e8,h5-e8,b8-e8,h8-e8,e5-e8}]

Plan (a) works for black, as long as the king is somewhere along the
green arrows and has a path for the pawn when white makes the
move. When so, the king can still catch the pawn
after \movecomment{1... exf5.}

For plans (b) and (c) to work for black, the king has to be able to
catch the breaking pawn. Thus, black king can be in any of the blue
squares after \movecomment{1. f5} and still catch the queener,
provided there is a clear path.

Combining, since black can choose the plan, the king has to be one of
the green or blue squares to prevent queening when it is white to move:

\fenboard{8/4k1p1/4p1Pp/4P2P/5P2/8/8/4K3 w - - 1 65}
\chessboard[clearfields={c5,e7},
  color=blue!30, pgfstyle=color,
  colorbackfields={d4,e4,g4},
  color=green!30,
  colorbackfields={b5,c5,d5,f5,g5},
  colorbackfields={b6,c6},
  colorbackfields={b7,c7,d7,e7},
  colorbackfields={b8,c8,d8,e8,f8,g8,h8},
  color=red!30,
  colorbackfields={h4}]

The only square for the king where the path becomes a problem is
h4. Consider \movecomment{1. f5 Kxh5 2. fxe6 Kxg6} and now the doubled
pawns prevent the black king from intercepting. On the other hand, the
white pawns cannot advance, either.

\fenboard{8/6p1/4P1kp/4P3/8/8/8/4K3 w - - 0 67}
\chessboard[clearfields={e1}]



\end{document}
