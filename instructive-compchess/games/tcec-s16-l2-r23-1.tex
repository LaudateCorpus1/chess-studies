%% skiminki's computer chess studies
%% Copyright (C) 2019 Sami Kiminki
%%
%% This program is free software: you can redistribute it and/or modify
%% it under the terms of the GNU General Public License as published by
%% the Free Software Foundation, either version 3 of the License, or
%% (at your option) any later version.
%%
%% This program is distributed in the hope that it will be useful,
%% but WITHOUT ANY WARRANTY; without even the implied warranty of
%% MERCHANTABILITY or FITNESS FOR A PARTICULAR PURPOSE.  See the
%% GNU General Public License for more details.
%%
%% You should have received a copy of the GNU General Public License
%% along with this program. If not, see <http://www.gnu.org/licenses/>.

\chessgame{Stoofvlees II a11}{chess22k 1.13}%
          {TCEC S16 League 2, Round 23.1}%
          {D05 - Colle System: Rubinstein Opening}%
          {2019-08-06}

Chess against a strong opponent can be a brutal endeavour, as black
found out in this game the hard way. Black started with some small
inaccuracies in the opening, which culminated in a strategic blunder
in the middle game. An engine as strong as Stoofvlees would not let
such opportunity pass.

Main points:
\begin{enumerate}
\item Opening discussion
\item Positional defensive weaknesses
\item Attacking tactics
\end{enumerate}

\mainline{1. d4 Nf6 2. Nf3 e6 3. e3 c5 4. Bd3 d5}

End of the opening book.

\mainline{5. b3 cxd4?!}

Black is perhaps releasing the tension a bit too eagerly. Now the dark
square bishop does not get an access to the c5 square, as would happen
if white could be persuaded to play \wmove{dxc5.} And indeed, the
mainline \movecomment{5... Nc6} scores significantly better than the
move played.\footnote{\movecomment{5... Nc6} 29\%-41\%-30\%/612
games; \movecomment{5... cxd4} 35\%-45\%-20\%/49 games; Lichess
masters database accessed on \printdate{2019-08-08}.}

\mainline{6. exd4 Nc6!?}

However, now after the exchange, perhaps better was to ask white to
make a slightly awkward pawn move with \variation{6... Bb4+ 7. c3}. The
pawn on c3 would block the bishop's vision, at least temporarily.

\mainline{7. Bb2 b6?!}

Much more common plays were \wmove{Bd6}, \wmove{Be7}, or \wmove{Bb4+.}
Pawn to \wmove{b6} superficially helps the light square bishop
development allowing \wmove{Bg7}. But it's not easy to see how black
could break the center to liberate the bishop. Perhaps better idea was
to play \wmove{Bd7}, instead, and \wmove{Bb5} later given a
chance. Further, as black still had the c-pawn anymore, the move would
have been sensible to support \wmove{c5}. But this is not the case
here.

\mainline{8. Nbd2\novelty}

\chessboard[
        pgfstyle=straightmove,
        color=blue, markmoves={c2-c4},
        pgfstyle=knightmove,
        color=blue,  markmoves={d2-c4},
        color=green, markmoves={d2-f3, f3-e5},
        color=red,   markmoves={d2-e4},
        color=red!30, pgfstyle=color, colorbackfields={e4}]

This is a flexible move:
\begin{enumerate}
\item The knight is ready to
hop in to f3 after the Nf3 knight moves to, say, e5 (green); and
\item Extra control is added on \wmove{d4} to discourage black's
potential \wmove{Ne4}, \wmove{f5} ideas (red); and
\item Support for the potential c4 push is added (blue).
\end{enumerate}

\mainline{8... Bb7 9. a3}

Restricting black's play by preventing \wmove{Ng4} and \wmove{Bg4.}
Also, preparing to meet black's a/b pawn pushes by adding the option
of fixing the queen-side pawns.

\mainline{9... Bd6 10. Ne5 O-O 11. Qe2 Nd7?!}

While seemingly logical in asking white what to do with \wmove{Ne5},
the problem here is that an important king-side defender is
displaced. White has pieces and pawns ready for an attack against the
king.

\mainline{12. O-O}

Castling supports the f-pawn push. That was probably the best
attacking idea available. Pawn to \wmove{f5} would begin to question
black's already weakened control of the center.

\mainline{12... Ncxe5 13. dxe5 Be7}

\wmove{Bc5} would have been met with immediate \wmove{b4} gaining a
tempo for white (\variation{13... Bc5 14. b4 Be7.}) \wmove{Bc7} would
have blocked the rook's access to c-file. With all likelyhood, the
best move was played.

\mainline{14. Nf3}

Now aiming for \wmove{Nd4}, blocking the d-pawn, and thus,
keeping \wmove{Bb7} off the play. Note that white's counterpart bishop
is significantly better, since it
can reroute itself via c1 if necessary, and it already supports the f-pawn
push nicely by protecting e5.

\mainline{14... Nc5 15. Nd4 Rc8}

Rooks belong to open files.

\mainline{16. f4 g6?!}

\chessboard[
        color=red!30, colorbackfields={f6,h6,g7}]

While not outright losing, it can be questioned whether black had to
weaken the king-side pawn structure, and particularly the dark
squares. In fact, this move can be considered as a strategic blunder,
as it opens new avenues for white's attack.

If black was afraid of \wmove{Bxh7+} or \wmove{Qh5}, the moves here to
play were either \wmove{Nxd3} or \wmove{h6.}

Stoofvlees was expecting \wmove{Rc7,} which adds defenses for
the \nth{7} rank and prepares for counterplay in the c-file. But
before playing \bmove{Rc7}, the move that had to be calculated
carefully was \wmove{Bxh7+.} But it turns out that the bishop
sacrifice would not have been mating, as black is just in time to
organize defenses. Variation \variation{16... Rc7 17. Bxh7+ Kxh7
18. Qh5+ Kg8 19. Rf3 Qe8 20. Rh3 f5 21. Qh8+ Kf7 22. Qh5+ Kg8 23. Qh7+
Kf7} would end peacefully.

\mainline{17. Rf3 a6}

White wastes no time, but the same cannot be said about black. Black
either needed to start diluting and preparing for the attack by,
e.g., \wmove{Nxd3}, \wmove{Kh8}, and \wmove{Rg8} to avoid pins and
putting counter-pressure on the g-file; and/or start preparing active
counterplay with \wmove{Rc7} intending to make something happen on the
semi-open c-file. \wmove{Rc7} would also add a defender on the nth{7}
rank provided \wmove{Be7} moves somewhere with \wmove{f5.}

\mainline{18. h4}

Stoofvlees offers a pawn in hope for opening files for attack. With
this Greek gift that should not be accepted, Stoofvlees's evaluation
jumped a bit. However, Stockfish suggests that black is still holding
with \wmove{Nxd3.}

\mainline{18... Bxh4?}

\chessboard

Black was too greedy. White has now the h-file available with strong
attack potential, and an engine as strong as Stoofvlees will not miss
the opportunity.

\mainline{19. Rh3! Be7}

Pawn to \wmove{g4} is coming, so the bishop needed to run either now,
or a concrete plan was needed to meet the follow-up move \wmove{Qh2.}
But regardless, there is no more defense anymore for black if white
plays precisely.

An example line: \variation{19... Rc7 20. Qg4 Nxd3 21. cxd3 Be7
22. Rh6 Kg7 23. f5! exf5 24. Qh3 Rh8 25. e6! f6 26. Rf1 Kg8 27. Nxf5!
Bc5+ 28. Nd4 Rg7 29. b4 Be7 30. Ne2 a5 31. Qe3 Bc8 32. Bd4 axb4
33. axb4 Qd6 34. Bxf6 Qxe6 35. Qd4 Bf8 36. Rh4 h5 37. Nf4 Qd6 38. Be5
Qxb4 39. Qxd5+ Rf7 40. Bxh8 Qc5+ 41. Bd4 Qxd5 42. Nxd5} and white wins
easily with an extra rook.

\mainline{20. g4 Nxd3 21. cxd3 b5 22. f5 Bg5}

The position is already desperate. While allowing the bishop to get
trapped with \wmove{f6,} the alternatives were not much better. For
example:

\begin{enumerate}[label=(\alph*)]
\item \variation{22... exf5 23. Qh2 h5 24. gxf5 Bg5 25. Kh1 Qc7
26. Qg2 Qe7 27. Rg1.} The attempt to save the bishop will end quickly:
\movecomment{27... f6 28. exf6 Bxf6 29. Qxg6+} with forced mate in 13.
\item \variation{22... Bc5 23. f6 Re8 24. Qh2 h5
25. Qf4 Qb6 26. Rxh5} with mate in 10.
\end{enumerate}

\mainline{23. Qh2 h5 24. f6 Bh6}

\chessboard

\mainline{25. Rxh5!}

Nice finishing touch. With the best play, mate would follow in 12 more
moves. The final idea here is to force the queen to g7 with an
unstoppable mate. The bishop can be dealt with g5.

\mainline{25... gxh5}

\variation{25... Be3+} would have postponed the inevitable by one
move. The most resilient continuation was \movecomment{26. Kf1 gxh5
27. Qxh5 Qa5 28. b4 Qc7 29. Re1 Bf4 30. g5 Bxe5 31. Ne2 d4 32. Qh6
Bg2+ 33. Kf2 Bg3+ 34. Kxg2 Qc6+ 35. Kh3 Qg2+ 36. Kxg2 Bxe1 37. Qg7#}


\mainline{26. Qxh5 Bf4 27. Rf1}

Faster was to immediately cut the bishop from defending
with \movecomment{27. g5} with mate in 6.

\mainline{27... Qa5 28. b4 Qa4 29. g5 Rc1
30. Bxc1 Bh2+ 31. Qxh2 Qxb4 32. axb4 Bc6 33. Qh6 Ba8 34. Qg7#}

White wins.
