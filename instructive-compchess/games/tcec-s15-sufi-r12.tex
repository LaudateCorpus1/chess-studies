%% skiminki's computer chess studies
%% Copyright (C) 2019 Sami Kiminki
%%
%% This program is free software: you can redistribute it and/or modify
%% it under the terms of the GNU General Public License as published by
%% the Free Software Foundation, either version 3 of the License, or
%% (at your option) any later version.
%%
%% This program is distributed in the hope that it will be useful,
%% but WITHOUT ANY WARRANTY; without even the implied warranty of
%% MERCHANTABILITY or FITNESS FOR A PARTICULAR PURPOSE.  See the
%% GNU General Public License for more details.
%%
%% You should have received a copy of the GNU General Public License
%% along with this program. If not, see <http://www.gnu.org/licenses/>.

\chessgame{LCZero v0.21.1-nT40.T8.610}{Stockfish 19050918}%
          {TCEC S15 Superfinal, Game 12}%
          {C05 French, Tarrasch, Closed}%
          {2019-05-12}


Following concepts exemplified:
\begin{enumerate}
\item Importance of the center to shelter the king when the king is
not behind pawns
\item Tactical sequences for positional gains
\item Positional analysis to create a fortress
\item Advanced endgame techniques: deflection and skewer
\end{enumerate}


\mainline{1. e4 e6 2. d4 d5 3. Nd2 Nf6 4. e5 Nfd7 5. f4 c5 6. c3 Nc6
7. Ndf3 Qb6 8. g3 Be7}

The most popular continuation here is \variation{8... cxd4 9. cxd4
Bb4+ 10. Kf2 g5 11. fxg5 Ndxe5 12. Nxe5 Nxe5 13. Kg2 Nc6 14. Nf3 Bf8
15. b3 Bg7 16. Bb2 Bd7}.

\mainline{9. Kf2}

\chessboard[pgfstyle=straightmove,
  color=yellow, markmoves={b6-f2}]

End of the opening book. This is a prophylaxis to avoid \wmove{Bb4}
with a tempo after \movecomment{... cxd4 cxd4}. See the mainline
\movecomment{8... cxd4} for details. But this move is not completely
without drawbacks, and a potential pin has to be considered.

In positions such as this, where the king is not safely behind the
pawns after the usual castling, trying to maintain a strong blockaded
center is often a wise decision. The reason is quite simple: without
the central pawns, black would be able to start attacking the white
king using diagonal through the center, often with double attacks of
forks and mate-threatening tactics. The importance of the central
pawns is exemplified by the queen on b6 eyeing the king.

However, since the center is not yet solidified, white has to be
careful. For instance, the e5-pawn cannot rely on the protection of
the d4 pawn, since after \movecomment{... cxd4 cxd4}, the white
d4-pawn would be pinned. This is not a problem right now, since the
e5-pawn is protected by the f4-pawn and the knight on f3. But, the
potential pin has to be constantly factored in when calculating the
responses to black's attempts to undermine the center with a typical
plan of f6.

A slightly more popular move than \movecomment{9. Kf2} was to
play \movecomment{9. Bh3,} instead. This discourages \bmove{f6} ideas
by exposing the resulting weakness in the e6-pawn.


\mainline{9... a5 10. a4 cxd4!}

This logical move has many upsides. First, it allows installation of a
piece in b4, as c3 will no longer control it. Second, the pawn on d4
becomes a bit weak. Third, white has to spend a tempo in order to move
the queen out of the pin soon. Fourth, the c-file is opened, which
should favor black due to white's king safety issues if black is able
to use by the rooks.

\mainline{11. cxd4 Ndb8}

\chessboard[pgfstyle=knightmove,
  color=green, markmoves={b8-a6,a6-b4},
  color=yellow, markmoves={c6-b4},
  color=red!30, pgfstyle=color,
  colorbackfields={d4}]

With a plan of \wmove{Nb8-a6-b4} installing an extra strong knight on
b4. If white takes, there's another knight ready to step in. Potential
weakness on d4 highlighted.

\mainline{12. Ne2 O-O 13. Kg2}
Now avoiding the pin, simplifying white's play in response
to \bmove{f6}. An interesting alternative would have been to
play \wmove{Bh3} first, allowing white to put some pressure on the
pawn on e6, discouraging \bmove{f6} by positional arguments against
the weakened e6.

\mainline{13... Na6 14. Nc3?!}
Here the validity of the move by white can be questioned. While there
are certainly ideas of playing \wmove{Na2} to challenge the b4 square
and \wmove{Nb5} blocking black queen's access to the b-file, more
importantly, the move also undermines the protection of the d4
pawn. This allows black to execute a better version of the f-pawn
push, forcing white to recapture with the d-pawn instead of the
f-pawn. White could have considered playing \wmove{Rb1}
and \wmove{Be3} first to solidify the d4 pawn, and only then
playing \wmove{Nc3}.

\mainline{14... f6! 15. h4 fxe5 16. dxe5}

\chessboard[pgfstyle=straightmove,
  color=green, markmoves={d5-d4}]

Now the argument against \movecomment{14 Nc3} has been made.
Capturing with the f-pawn would now have allow black sacrificing the
exchange for a knight and pawn as a direct
consequence: \variation{16. fxe5 Rxf3! 17. Qxf3 Nxd4 18. Qd1} with
plenty of compensation. Also, the earlier \movecomment{15. exf6 Bxf6}
would not have been attractive either, as black would be able to put
proper pressure on the pawn on d4. So, white was forced to capture
with the d-pawn, and black has now a scary-looking protected potential
passer on d5, which is now controlling important squares c4 and e4.

\mainline{16... h6 17. Bd3 Nab4}
Black has now finally executed the plan to install an extra strong
knight on b4. Sometimes, it is said that knights protecting each other
are clumsy, because they're in each other's way. And surely, in
endgames, this can be true, especially when the knights are the only
thing protecting each other. However, this is different, since the
knight on c6 serves as a replacement in case the knight on b4 is
captured.

The knight on b4 is ready to support push of the d-pawn up to d3.

\mainline{18. Bb1 Bd7 19. Na2 Rac8 20. Kh3}
\chessboard[pgfstyle=straightmove]

This little move deserves special attention. While \movecomment{20. Kh3} may
look like a prophylaxis, and it is, it also puts the king in the same
diagonal with
\wmove{Bd7.} This makes e6 pawn a bit less weak, since white has to spend an
extra tempo to move the king away from the diagonal before e6 can be
captured without a pin. However, with the king safety being somewhat
questioned, being sheltered by an enemy pawn is probably better than
leaving the king in g2 awaiting for tactics. After the knight moves
away from c6, the king on g2 would be subject to \bmove{d4} with a
follow-up check through the a8-h1 diagonal with tempo gains.

\mainline{20... Be8 21. Nxb4 Nxb4 22. Ra3 Bc5 23. Bd2 Qa6 24. Rc3}

White is not quite in time to defuse black's pressure. If given a free
tempo, say \variationcurrent{24... Kh8 25. Be3 b6 26. Nd4 Bxd4
27. Bxd4}, and white would be able to blockade the d-pawn and perhaps
starting to target the pawn on b6 or preparing \wmove{g4}
and \wmove{f5} with the idea to create a passer on the
e-file. However, tempi are a scarcity in chess.

Instead of trying to keep the tension with \movecomment{24. Rc3,}
maybe it was time to relieve the tension a bit
with \variation{24. Bxb4} and go for a draw.

\mainline{24... b5}
\chessboard[pgfstyle=straightmove]

One thing that always amazes is how the computers so casually allow
pins and leave pieces hanging. But of course, the computers are able
to calculate through tactics. Many humans would understandably start
looking into solidifying moves such as \wmove{Bd7} to
prevent \wmove{Qc1} pinning the bishop on c5. Instead of solidifying
the position and trying to untangle the pins, black complicates the
position to win a pawn.

\mainline{25. Qc1}

Taking the b-pawn would be problematic: \variation{25. axb5?!
Bxb5\withattack} and white would have annoying threats such
as \bmove{Be2} fork to deal with. Pinning the bishop was the better
choice, although black can strenghten the protection of the bishop on
c5 just in time.

\mainline{25... Qb6 26. Nd4 bxa4}

This move was the point of the tactical complications
of \movecomment{24... b5}, changing the queen-side pawn structure to
favor black. White is now left with a weak b-pawn against doubled
a-pawns and a passer on d5.

\mainline{27. Be3 Bd7}

Now black allows a tactical sequence by white winning an exchange. If
black wanted, the next move could have been prevented
by \variation{27... Rc7.} White enters now in a forced sequence.

\mainline{28. Nf5!? Bxe3!}

Note that giving up the exchange early with \variation{28... Rxf5
29. Rxc5 Rxc5} would have given white two pleasant options:
(1) \movecomment{30. Bxc5 Qa6 31. Bxf5 exf5\wupperhand} taking the
exchange with roughly equal pawn structure, as white also has a
passer; or (2) \movecomment{30. Qxc5 Qxc5 31. Bxc5 Rf7 32. Bg6 Bb5
33. Bxf7+ Kxf7\wupperhand} delaying the taking of the exchange a bit,
keeping the pawn structure but simplifying the positions with
exchanges. Both variations are likely winning for white.

\mainline{29. Ne7+ Kf7 30. Nxc8 Rxc8 31. Qxe3 d4}

Forcing sequence ends. Here white could have also untangled from the
fork by \movecomment{32. Qd2,} as \movecomment{32... dxc3??} would
fall for \movecomment{33. Qxd7+ Kf8 34. Qxc8+} with mate soon to
follow.

\mainline{32. Qf2 Qb7 33. Rcc1 d3}

\chessboard[pgfstyle=straightmove]

Now black has finally been able to push the d-pawn to d3. Since the
pawn is protected by the strong knight on b4 and black can enforce the
protection by the light square bishop, white has no good way to
challenge the pawn on d3. Note that the white king cannot come to
assist in the capture either, because after the exchanges, white's
king-side pawns would be subject to be captured by the black king.

We are now going to fast forward to the next critical position.

\mainline{34. h5 Rxc1
35. Rxc1 Qd5 36. Kh4 Bc6 37. g4 Ke8 38. Rf1 Qd8+ 39. Kg3 Qd5 40. Rd1
Qb3 41. Rd2 Qc4 42. Rd1 Qb3 43. Qd2 Be4 44. Re1 Bb7}

\chessboard[pgfstyle=straightmove]

The pawn move \wmove{f5} here is interesting, and it seems a way for
white to force a draw, although the play is not forced by either
side. For example, \variationcurrent{45. f5 exf5 46. gxf5 Qd5 47. Re3}
and black can still hang on to the d3 pawn
with \movecomment{47... Bc8} with the idea of \movecomment{48. Bxd3??
Bxf5!}. But after \movecomment{48. e6 Qxf5 49. Bxd3} it would be black
who has to be careful.

An interesting alternative play for black would
be \movecomment{46... Nc2}. White has no oblication to take on c2, but
it would lead in a nice way to force a draw. \movecomment{47. Bxc2
dxc2+ 48. Re3 Qxb2 49. f6.} Here black could promote to get another
queen, but white would be just in time with \movecomment{49... c1=Q
50. f7+ Kxf7 51. Qd7+ Kg8 52. Qe8+ Kh7 53. Qg6+ Kg8 54. Qe8+}, drawing
with perpetual checks.

\mainline{45. Rd1 Be4
46. Kh4 Qd5 47.  Qc3 Kf7 48. Qc7+ Kg8 49. Qc3 Kh7 50. Kh3 Kg8 51. Kh4
Kh7 52. Kg3 Kg8 53. Qc8+ Kh7 54. Qc1 Qb3 55. Qd2 Kg8 56. g5}

\chessboard[pgfstyle=straightmove]

Taking the pawn on g5 here might be a small
inaccuracy: \variationcurrent{56... hxg5 57. fxg5 g6} to prevent white
from playing g6 himself with backrank mate threats. An example
contituation: \movecomment{58. Rc1 Nc2 59. Kf4 Qb7 60. Bxc2 dxc2
61. Qd8+ Kh7 62. Qf8 gxh5 63. Qh6+ Kg8 64. Qxe6+}, but this should
still be about equal.

\mainline{56... Kf8 57. g6 Ke7 58. Kh3 Ke8 59. Kh4 Qc4 60. Kg3 Bf5
61. Qc3}

\chessboard[pgfstyle=straightmove]

This is a committal move, offering the queen exchange, which black
accepts. While the game has been objectively close to a draw and still
is, only black can press for the win. With queens on the board, there
was always a possibility for some dynamic play.

\mainline{61... Qxc3 62. bxc3}

It is possible that white thought that at this position, the d3-pawn
could be somehow won, possibly by giving the exchange back. If that
was the case, then white king could hold or take the black pawns on
the a-file. However, the d-pawn can never be taken without losing the
game.

\mainline{62... Nc2}

\chessboard[
  color=red!30, pgfstyle=color,
  colorbackfields={e1,e2,e3,e4}]

This knight now becomes a very annoying piece. Together with the pawn
and the bishop, all white king's access squares to the d-pawn are
controlled.

\mainline{63. Kf2}
White now has a threat of \wmove{Bxc2}, and white would be just in
time to stop the black pawns.

\mainline{63... a3!}
Parries the threat of Bxc2. Now black king is just in time to escort
the a-pawns.

\mainline{64. Ba2}
Taking the c2-pawn would have been a huge blunder. \variation{64. Bxc2
dxc2 65. Rc1 Kd7 66. Ke3 Kc6 67. Kd2 Kb5 68. Rxc2 Kc4!} Taking the
rook with Bxc2 would have been a huge blunder, since white has a
winning king-side pawn break. \movecomment{69. Rc1 Kb3 70. Re1 a2} and
there's no way of stopping \wmove{a1Q} other than giving up the
rook.


\mainline{64... Kd7 65. Bb3 Kc7 66. Rg1 Kd7}

\chessboard

This is the final position in the game where the game was still
objectively a draw.

\mainline{67. Kf3?}
Computer analysis suggests that white had at seven moves which would
have maintained the draw. Unfortunately, the move played was none of
them. The problem with \movecomment{67. Kf3?} is that it
allows \movecomment{67... d2} with precise tactics as played, winning
the game.

The key for white maintaining the draw is to set up a dynamic
fortress, preventing black's progress. Let us take a closer look.

Perhaps the easiest way to set up the defenses is the straightforward
Ba4+.

\movecomment{67. Ba4+ Kc7 68. Bb3.} The bishop on b3 and the
pawn on c3 guard the entry squares for the black king, and the bishop
additionally stops the immediate a2 and a4. The rook's job is to
create enough harassment to prevent the black bishop to enter a square
to protect the d1 promotion square, and the d-pawn push. Black would
need two tempi to prepare \wmove{d1=Q}, but will never have enough
time with the best defense.

\movecomment{68... Kc6 69. Rd1.} This is the easiest
plan. Now \wmove{Bg4} is prevented, as d3 would be hanging.

\movecomment{69... Kb5 70. Rb1.} The only
move. Black has to move the king away to prevent Bxc2+ exposure check,
and thus, \wmove{a4} or \wmove{Bg4} here is prevented.

Black has to be careful not to overextend, and thus has to
retreat. The c5-square is off limits for the king here, since that
would allow the white rook to enter the 8th rank. For example:
\movecomment{70... Kc5? 71. Ba2! Kc6. 72. Rb8}, and the white rook would
start picking up the black pawns. There is also no time for black to
play \wmove{d2} in this line, since the rook would simply move to the
d-file to pick up the pawn.

\movecomment{70... Kc6 71. Rd1} to prevent \wmove{Bg4}. Black has no
way to make progress. The final attempt is to play \wmove{a4} after
{Ba2}.

\movecomment{71... Kc7 72. Ba2 a4.}  However, \wmove{Ba2}
and \wmove{c3} will control the entry squares for the king, and
without the king, the d-pawn can never promote successfully.

\mainline{67... d2! 68. Ke2 Ne1!}

The d2-pawn is untouchable, \movecomment{69. Kxd2 Nf3+} and black
picks up the rook.

\mainline{69. Kd1 Nf3}

The pawn on d2 is now protected. However, precise play is still needed
for conversion, but that is no problem for Stockfish.

\mainline{70. Rh1 Bd3}

\chessboard

Black's threat here is to play \wmove{Bb1} and then deflect the white
bishop with \wmove{a4}, and when the bishop is no longer controlling
the a2-square, then play \wmove{a2} and \wmove{a1=Q}. But that has to
be prepared by moving the queen out of \wmove{Bxa4+} check.

\mainline{71. Ba4+ Ke7 72. Bb3 Bb1}
Now the \wmove{a4} deflecting threat is enabled. Now, white is really
out of moves.

\mainline{73. Rh3}

Another try was \variation{73. Rf1 a4 74. Bc4 Bd3! 75. Bxc3 a2,} but
that does not work, either.

\mainline{73... Be4 74. Rh1 Nxe5}

Picks up a pawn. White cannot give the exchange
back: \variationcurrent{75. fxe5 Bxh1 76. Kxd2 a4 77. Ba2 Bf3 78. c4
Bxh5 79. Kc3 Bxg6 80. c5 Be4 81. Bc4 h5} and white cannot stop both a
and h-pawns.


\mainline{75. Rf1 a4 76. Ba2}
\variation{76. fxe5 axb3} with connected unstoppable passers for black.

\mainline{76... Nf3 77. Rh1 Bd3 78. Rh3 Bf5 79. Rh1}

The knight cannot be taken due to skewer. \variation{79. Rxf3 Bg4
80. Ke2 Bxf3+ 81. Kxd2} and black picks up white's h5 and g6 pawns.

\mainline{79... Kd6 80. Bc4 Bb1 81. Rh3}

\chessboard

0-1. Black wins by adjudication.

A possible continuation: \variationcurrent{81... Bd3 82. Ba2 Be2+
83. Kxe2 Ng1+ 84. Kxd2 Nxh3 85. Ke3 e5 86. f5 Nf4 87. f6 Nxh5 88. f7
Ke7 89. Ke4 Nf4 90. Kxe5 Nd3+ 91. Ke4 Nc1 92. Bb1 a2 93. Bxa2 Nxa2
94. Kd3 Nc1+ 95. Kd2 Nb3+ 96. Kc2 h5}. White king cannot stop both a
and h-pawns.


\chessgameappendix{}

\fenboard{8/6p1/4p1Pp/2k1P2P/5P2/8/8/4K3 b - - 0 1}
\chessboard[clearfields={c5,e1}]

When evaluating transitions to endings, it is often useful to analyze
specific pawn structures on the board and whether they're winning or
not given the remaining pieces.

The pawn structure in the figure is winning for white, unless there is
a black piece to stop the queening. The plan for white is to play f5,
and then:
\begin{enumerate}[label=(\alph*)]
\item if black takes, push the e-pawn. Full
variation: \movecomment{1. f5 exf5 2. e6 f4} and the e-pawn runs.
\item if black doesn't take, push the f-pawn again. Full variation:
\movecomment{1. f5} and \movecomment{2. f6 gxf6 3. g7} and the g-pawn
queens on the next move.
\item if black still doesn't take, take the g-pawn with the
f-pawn. Full variation: \movecomment{1. f5} and \movecomment{2. f6}
and \movecomment{3. fxg7} and the g-pawn queens on the next move.
\end{enumerate}

Now, when considering a king-pawn ending with this structure, it is
quite straightforward to determine the area where the black king has
to be in order to stop the queening, provided white king is far
away. Consider that \movecomment{1. f5} has been played.

\fenboard{8/6p1/4p1Pp/2k1PP1P/8/8/8/5K2 b - - 0 68}
\chessboard[clearfields={c5,e1},
  color=blue!30, pgfstyle=color,
  colorbackfields={d4,e4,f4,g4,h4},
  pgfstyle=straightmove,
  color=green, markmoves={b5-e8,h5-e8,b8-e8,h8-e8,e5-e8}]

Plan (a) works for black, as long as the king is somewhere along the
green arrows and has a path for the pawn when white makes the
move. When so, the king can still catch the pawn
after \movecomment{1... exf5.}

For plans (b) and (c) to work for black, the king has to be able to
catch the breaking pawn. Thus, black king can be in any of the blue
squares after \movecomment{1. f5} and still catch the queener,
provided there is a clear path.

Combining, since black can choose the plan, the king has to be one of
the green or blue squares to prevent queening when it is white to move:

\fenboard{8/4k1p1/4p1Pp/4P2P/5P2/8/8/4K3 w - - 1 65}
\chessboard[clearfields={c5,e7},
  color=blue!30, pgfstyle=color,
  colorbackfields={d4,e4,g4},
  color=green!30,
  colorbackfields={b5,c5,d5,f5,g5},
  colorbackfields={b6,c6},
  colorbackfields={b7,c7,d7,e7},
  colorbackfields={b8,c8,d8,e8,f8,g8,h8},
  color=red!30,
  colorbackfields={h4}]

The only square for the king where the path becomes a problem is
h4. Consider \movecomment{1. f5 Kxh5 2. fxe6 Kxg6} and now the doubled
pawns prevent the black king from intercepting. On the other hand, the
white pawns cannot advance, either.

\fenboard{8/6p1/4P1kp/4P3/8/8/8/4K3 w - - 0 67}
\chessboard[clearfields={e1}]

