%% skiminki's computer chess studies
%% Copyright (C) 2019 Sami Kiminki
%%
%% This program is free software: you can redistribute it and/or modify
%% it under the terms of the GNU General Public License as published by
%% the Free Software Foundation, either version 3 of the License, or
%% (at your option) any later version.
%%
%% This program is distributed in the hope that it will be useful,
%% but WITHOUT ANY WARRANTY; without even the implied warranty of
%% MERCHANTABILITY or FITNESS FOR A PARTICULAR PURPOSE.  See the
%% GNU General Public License for more details.
%%
%% You should have received a copy of the GNU General Public License
%% along with this program. If not, see <http://www.gnu.org/licenses/>.

\chessgame{Ethereal 10.88}{Lc0 17.11089}{CCCC 1: Rapid Rumble
  (15|5)\\Stage 1 Round 35}{B90 Najdorf, Byrne (English) attack}%
          {2018-09-10}

Following concepts exemplified:
\begin{enumerate}
\item Prying lines open with pawn-and-piece attacking pawn moves
\item The triangle of interception for pawn and king races
\item Blocking two pawns with a knight
\end{enumerate}

The engines started play from the usual start position.

\mainline{1. e4 c5 2. Nf3 d6 3. d4 cxd4 4. Nxd4 Nf6 5. Nc3 a6}

The popular Najdorf variation of the Sicilian defence.

\mainline{6. Be3}

The Byrne (English) attack.

\mainline{6... e5 7. Nb3 Be6 8. f4}

The first move to diverge from the mainlines. The move is not bad at
all, but makes the game sharper than the most popular move,
\movecomment{8. f3.}

\mainline{8... exf4 9. Bxf4 Nc6 10. a3?!\novelty}

\chessboard

This is a novelty.\footnote{Lichess masters database, accessed Apr
  2019.} The main moves here are \movecomment{10. Qe2} and
\movecomment{10. Qd2,} preparing for long castling. This seemingly
unnecessary move has some potential issues:

\begin{enumerate}
\item Black can play Bxb3 and white has to take with the c-pawn,
  instead of having the additional option to take with the a-pawn.
\item Black has later additional options to open Q-side files by
  pushing the b-pawn.
\end{enumerate}

\mainline{10... Be7 11. Qd2 Nh5 12. Be3 Nf6 13. O-O-O O-O 14. Kb1 b5}
\chessboard[color=red, pgfstyle=straightmove, markmoves={b5-b4}]

Now exercising black's typical Najdorf plan after opposite-side
castling. The threat is now to push \wmove{b4} and force \wmove{axb4}.

\mainline{15. Nd4 Nxd4 16. Bxd4 Rb8 17. Be2 Nd7 18. Rhf1 Bg5 19. Qd3 Qe7 20. Bf2 b4}
\chessboard

Now finally pushing b4. This is a typical attacking move to pry open
files. If the knight moves, \wmove{bxa3} will be played.

\mainline{21. axb4 Rxb4 22. Qxd6 Rfb8 23. b3 Qd8 24. Nd5 Rxe4 25. Rfe1
  Bf5 26. Kb2 Rc8 27. Bd3 Rxe1 28. Bxe1 Bg4 29. Ra1 Nc5 30. Qxd8+
  Rxd8}

\chessboard

White has managed to temper black's attack. The position is still
somewhat complicated, but should be objectively a draw.

\mainline{31. Bc4 Be6 32. Rd1 Rd7 33. Bb4 Ne4 34. Kb1 Bd8 35. Rd4 Nf2
  36. Bc5 Ng4 37. Nb4 Rxd4 38. Bxd4 Bxc4 39. bxc4 Bf6?!}

\chessboard

This is a dubious-looking move, offering extra options for white to
drive the endgame into a desireable direction. The game should still
be a draw. Black had a number of other options, such as
\movecomment{39... Nxh2} taking the pawn, \movecomment{39... f5}
hastening the pawn push, or \movecomment{39... Kf8} bringing the king
to play. But since black played \movecomment{39... Bf6?!}, white has
at least three options to choose from:

\emph{Option 1 --- Deflect the knight from attacking h2, gaining a bit
  of time.}  \variationcurrent{40. Bxf6 Nxf6.}

\emph{Option 2 --- Exchange the bishop for the knight, simplifying the
  position.}  \variationcurrent{40. h3 Bxd4 41. hxg4}. While this
position may look a bit weird at first, white's pawns are all in the
light squares, untouchable by the black bishop, and white has double
passed c-pawns while black's a-passer is weak.

But white decided to go with the third option: \wmove{Bg1.} This move
has multiple purposes: (1) it protects h2; (2) it protects the next
square for the c4-pawn; and (3) it avoids exchanging pieces. The
dark-square bishop alone cannot support the advancement of the c-pawn,
and the knight alone is clumsy. But the bishop and the knight
generally work well together to control squares on the way of the
pawn. This was also the best choice for white.

\mainline{40. Bg1 Ne5 41. c5}

Now is the time for black to play \wmove{Kf8}. The black king needs to
move in time to stop the c5 pawn. Highlighted is the triangle of
interception where the king needs to be after black's move in order to
win the race.

\chessboard[color=blue, pgfstyle=straightmove, markmoves={f8-c5, f8-c8},
  color=red, markmoves={c5-c8},
  color=yellow!50, pgfstyle=color, backareas={c8-c5, d8-d6, e8-e7, f8-f8},
  pgfstyle=straightmove, color=black, markmove={g8-f8}]

However, black did not play \wmove{Kf8} and black pieces alone cannot
block or win the passed pawn. Preventing \wmove{c8=Q} will now cost a
piece.

\mainline{41... a5?? 42. c6 Bd8 43. Bd4 Bc7}

Black at least gets now two pawns for the piece. Note that
\variation{43... axb4 44. Bxe5 Kf8 45. c7 Bxc7 46. Bxc7} would have
been even worse for black.

\mainline{44. Bxe5 Bxe5 45. Na6 Bxh2 46. c7 Bxc7 47. Nxc7 f5}

\chessboard

Here Leela thinks she is somewhat better with black, and this
misevaluation was probably the reason why Leela played the losing
\bmove{41... a5??} move earlier. A beginning player might make a
similar evaluation mistake.

The king-side black pawns surely look intimidating, but this is only
superficial. If one does not calculate and/or spot the pattern to stop
the king-side pawns, it is plausible to think that black has time to
march the king to support the a5-pawn, block white's passed c-pawn,
and overrun the king-side with the pawn wall. If white king intercepts
this plan, surely the white knight and a pawn cannot stop the black
pawn wall?

However, white has an easy plan. The g2 pawn guarantees that at least
one black pawn will be exchanged when the black pawns march
forward. If the knight is in time, it can stop two black pawns with
ease with the L-shaped defensive pattern. In fact, white even has the
time to take a small detour with the knight and take the a5, and then
to blockade the king-side pawns.

When choosing the blockading squares for the knight, it is here
beneficial to block the black pawns as late as possible, because that
would require the black king to move beyond the c4 pawn for support,
and then c4 pawn would have easy time to march forward and queen.

This plan is a forced win for white with the best but not difficult
play.

\mainline{48. Ne6 Kf7 49. Nd8+ Kf6 50. Nb7 a4 51. Nc5 g5 52.
  Nxa4 h5 53. Nc5 h4 54. Kc1 g4 55. Nd3 h3 56. gxh3 gxh3}

As promised, the g2 pawn was able to take one black pawn from the
wall. Two pawns left for black.

\mainline{57. c3 h2 58. Nf2}

\chessboard[pgfstyle=knightmove, color=blue, markmoves={f2-h1, h1-f2}]

Now the knight controls the h1 and f2 squares and the black pawns
cannot advance through those squares without support. However, the
king cannot offer assistance, since the c4-pawn would run.

Further, it is important to note that the knight can jump freely
between the h1 and f2 squares to lose tempi if necessary. Losing (or
gaining) tempi is often important in king-pawn endings, and this
ending is not an exception. With only the white king and pawn versus
black king, this would be an easy draw.

\mainline{58... Ke5 59. Kd2 Kd5 60. Kd3 Ke5 61.
  Ke3 Kd5 62. Nh1}

\chessboard

The first tempo loss, basically asking black to make another
move. However, \movecomment{62. Kd3} and losing the tempo later was
equally good.

Black has here one last attempt to trick a draw, although this variant
was not played: \variationcurrent{62... f4+ 63. Kxf4?? Kc4 64. Kg3
  Kxc3 65. Kxh2} draw. The correct move for white was
\movecomment{63. Kd3!} letting the knight to stop the f-pawn, and not
allowing the king to stray away from the all-important c-pawn.

\mainline{62... Kc4 63. Kd2 Kd5 64. Kd3 f4 65. c4+ Kc5 66. Kc3 Kd6
  67. Kd4 Kd7 68. c5 Kd8 69. c6 Kc8 70. Kc5 Kc7 71. Kb5 f3}

\chessboard

Black does not want to move the king, as this allows white to push
forward with \wmove{Kc6}. But after the pawn moves are exhausted,
black is out of options.

\mainline{72. Kc5 f2 73. Nxf2 Kc8 74. Kb6 Kd8 75. c7+}

\chessboard

Without the knight and black pawn, \wmove{Kc8} would be a draw, since
the only way (\wmove{Kc6}) white can protect the pawn would lead in a
stalemate. But here white can always lose a tempo with a knight move,
and ask black to make another move. So, black simply gives up.

\mainline{75... Kd7 76. Kb7 Ke6 77. c8=Q+ Ke5 78. Qh3 Kf4 79. Qxh2+
  Kf3 80. Qh3+}

Final note. Ethereal follows the quickest distance to zero in
tablebase win positions. \movecomment{80. Kc6} would have delivered
the mate one move earlier.

\mainline{80... Kxf2 81. Kb6 Ke2 82. Kc5 Kf2 83. Kd4 Kg1 84. Kd3 Kf2
  85. Qg4 Kf1 86. Ke3 Ke1 87. Qg1\mate}

White wins.
